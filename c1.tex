\documentclass{matrix-homework}
% \usepackage{showframe}
\usepackage{extarrows}

\chapternumber{1}
\begin{document}

\section{}
试证:所有$n$阶实对称矩阵构成$\dfrac{n(n+1)}{2}$维线性空间;所有$n$阶反对称矩阵构成$\dfrac{n(n-1)}{2}$维线性空间。

\answer
显然, 在所有$n$阶实对称矩阵构成的空间上可以定义矩阵加法与数乘运算,其满足封闭性与其他各项基本性质,下面主要对其维数进行分析.

$n$阶实对称矩阵在其主对角线上有$n$个独立元素.对于主对角线以外的部分,其上三角与下三角元素有重复,只需要考虑一部分,即$\dfrac{n(n-1)}{2}$个独立元素.故该线性空间维数为
$$ n + \dfrac{n(n-1)}{2} = \dfrac{n(n+1)}{2} $$

类似地,反对称矩阵主对角线元素为0,其他部分有$\dfrac{n(n-1)}{2}$个独立元素,线性空间维数为
$$\dfrac{n(n-1)}{2}$$
\qed

\section{}
设$R[x]_4$是所有次数小于4的实系数多项式组成的线性空间,求多项式$p(x)=1+2x^3$在基$1,x-1,(x-1)^2,(x-1)^3$下的坐标。

\answer
设$t=x-1$,则$x=t+1$,从而有
\begin{align*}
    p(x) & = 1+2(t+1)^3                 \\
         & = 2(t^3 + 3t^2 + 3t + 1) + 1 \\
         & = 2t^3 +6t^2 +6t +3
\end{align*}
故坐标为$(3,6,6,2)$
\qed

\newpage
\section{}
\newcommand{\Axi}[1]{\underline{A}^{#1}\left(\xi\right)}
设$\underline{A}$是$n$维线性空间$V$的一个线性变换,对某个$\xi \in V$有$\Axi{k-1} \neq0, \Axi{k-} =0$, 试证: $\xi,\Axi{},\Axi{2},\cdots,\Axi{k-1}$线性无关

\answer
使用反证法,设$\xi,\Axi{},\Axi{2},\cdots,\Axi{k-1}$线性相关,于是有不全为0的系数$c_0,c_1,\allowbreak c_2,\allowbreak\cdots,c_{k-1}$使得
$$ c_0\xi + c_1\Axi{1} + c_2\Axi{2} + \cdots + c_{k-1}\Axi{k-1} = 0 $$
对上式两端同时进行线性变换可得
\begin{align*}
                                   & c_0 \Axi{1} + c_1 \Axi{2} + c_2 \Axi{3} + \cdots + c_{k-1} \Axi{k}      \\
    \xlongequal[]{\Axi{k}=0} \quad & c_0 \Axi{1} + c_1 \Axi{2} + c_2 \Axi{3} + \cdots + c_{k-2} \Axi{k-1}    \\
    =                        \quad & \underline{A}(0)    =                                                 0
\end{align*}
重复上述过程,最终得到
$$c_0\Axi{k-1}=0$$
由于$\Axi{k-1}\ne0$,所以$c_0=0$,将该结果向上一次变换得到的等式带入得到
$$c_0\Axi{k-2} + c_1\Axi{k-1}=0$$
故$c_1=0$.不断重复此步骤,最终得到$c_0=c_1=c_2=\cdots=c_{k-1}=0$,与原假设矛盾.

故原假设不成立,$\xi,\Axi{},\Axi{2},\cdots,\Axi{k-1}$线性无关.
\qed

\section{}
设$R[x]_4$中,试求在基$\alpha_1=1,\alpha_2=x,\alpha_3=x^2$下的矩阵为
\begin{equation*}
    \begin{bmatrix}
        1 & 1 & 1 \\
        0 & 1 & 1 \\
        0 & 0 & 1
    \end{bmatrix}
\end{equation*}
的线性变换$\underline{A}$,并求$g=2+4x-7x^2$的像.

\answer
$\underline{A}(x_0,x_1,x_2)=(x_0+x_1+x_2, x_1+x_2, x_2)$

$\underline{A}(g)=-1-3x-7x^2$
\qed

\section{}
已知两个$n$维列向量$\alpha=[a_1,a_2,\cdots,a_n]^{\TT},\beta=[b_1,b_2,\cdots,b_n]^{\TT}$都是非零列向量,$b_1\ne0$,且$\alpha^\TT\beta=0$,若$A=\alpha\beta^\TT$,求$A$的特征值与特征向量.

\answer
由$A=\alpha\beta^\TT$可知,$A$的秩为1,至多有一个非零特征值

又有$\Tr{A} = \Tr{\alpha\beta^\TT} = \alpha^\TT\beta=0$,$A$的全部特征值和为0,故A的所有特征值均为0.

对于特征值$\lambda=0$,其特征向量$v$满足
$$Av=\alpha\beta^\TT v=\lambda v=0$$
故$\beta^\TT v=0$,特征向量$v$是与$\beta$正交的任意非零向量.
\qed

\section{}
在线性空间$R[x]_4$中,基$1,x,x^2,x^3$到基$1,1-x,1-x-x^2,1-x-x^2-x^3$的过渡矩阵为.

\answer
记旧基底$1,x,x^2,x^3$为$\alpha_1, \alpha_2, \alpha_3, \alpha_4$,新基底$1,1-x,1-x-x^2,1-x-x^2-x^3$为$\beta_1, \beta_2, \beta_3, \beta_4$.
有
\begin{equation*}
    \left\{
    \begin{aligned}
        \beta_1 & = \alpha_1                           \\
        \beta_2 & =\alpha_1-\alpha_2                   \\
        \beta_3 & =\alpha_1-\alpha_2-\alpha_3          \\
        \beta_4 & =\alpha_1-\alpha_2-\alpha_3-\alpha_4
    \end{aligned}
    \right.
\end{equation*}
于是有
$$
    \begin{bmatrix}
        \beta_1 & \beta_2 & \beta_3 & \beta_3
    \end{bmatrix}
    =
    \begin{bmatrix}
        \alpha_1 & \alpha_2 & \alpha_3 & \alpha_3
    \end{bmatrix}
    \underbrace{
        \begin{bmatrix}
            1 & 1  & 1  & 1  \\
            0 & -1 & -1 & -1 \\
            0 & 0  & -1 & -1 \\
            0 & 0  & 0  & -1
        \end{bmatrix}
    }_{P}
$$
其中$P$即为过渡矩阵.
\qed

\newpage
\section{}
设$R[x]_4$是所有次数小于4的实系数多项式组成的线性空间,求多项式$p(x)=3+2x^3$在基$1,x-1,(x-1)^2,(x-1)^3$下的坐标。

\answer
设$t=x-1$,则$x=t+1$,从而有
\begin{align*}
    p(x) & = 3+2(t+1)^3                 \\
         & = 2(t^3 + 3t^2 + 3t + 1) + 3 \\
         & = 2t^3 +6t^2 +6t +5
\end{align*}
故坐标为$(5,6,6,2)$
\qed

\section{}
已知数域$F$上的线性空间$M_{3\times 3}(F)$,令$V=\{A\in M_{3\times 3}(F) | \Tr{A}=0\}$,则$V$的维数为\_\_\_\_,写出$V$的一组基\_\_\_\_.

\answer
$\mr{dim}(V)=8$,

$\{a_{12},a_{13},a_{21},a_{23},a_{31},a_{32},a_{11}-a_{22},a_{22}-a_{33}\}$(其中$a_{ij}$表示仅第$i$行$j$列为1的$3\times3$矩阵)
\qed

\section{}
在线性空间$R^3$中,定义线性变换$f([x_1,x_2,x_3])=[x_3,x_2+x_3,x_1+x_2+x_3]$,其中$x_i\in R$,则$f$在基$\varepsilon_1=[1,0,0]$,$\varepsilon_2=[0,1,0]$,$\varepsilon_3=[0,0,1]$下的矩阵表示为

\answer
$
    \begin{bmatrix}
        0 & 0 & 1 \\
        0 & 1 & 1 \\
        1 & 1 & 1
    \end{bmatrix}
$\qed

\section{}
在线性空间$R^3$中,我们取三个向量$\alpha_1=[1,-1,1]$,$\alpha_2=[2,1,0]$,$\alpha_3=[3,0,1]$,分别生成子空间$V_1=L(\alpha_1,\alpha_2)$,$V_2=L(\alpha_1,\alpha_3)$,则$V_1\cap  V_2$的维数为.

\answer
$\alpha_1+\alpha_2=\alpha_3$

$\Dim{V_1\cap V_2} = \Dim{V_1} + \Dim{V_2} - \Dim{V_1 + V_2} = 2+2-2 = 2$
\qed

\end{document}
