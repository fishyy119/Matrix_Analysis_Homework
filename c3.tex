\documentclass{matrix-homework}

\chapternumber{3}
\begin{document}
\makecover{3}

\section{}
举例说明可对角化的矩阵不一定是正规矩阵.

\answer
取两个线性无关但是不正交的向量,令

$P=[\alpha_1,\alpha_2]=\mat{1&1\\0&1}$,$D=\mat{1&0\\0&2}$

则$A=PDP^{-1}=\mat{1&1\\0&1}\mat{1&0\\0&2}\mat{1&-1\\0&1}=\mat{1&1\\0&2}$

验证从略,$A$具有两个不正交的特征向量$\alpha_1,\alpha_2$,不是正规矩阵
\qed


\section{}
设$A$为正定H矩阵,$B$为H矩阵,试证:$AB$与$BA$的特征值为实数.

\answer
由于$A$为正定H矩阵,所以存在可逆矩阵$Q$,使得$A=Q^\HH Q$,于是有
$$AB=Q^\HH Q B= Q^\HH QB Q^\HH (Q^\HH)^{-1} \sim QBQ^\HH$$

由于$B$是H矩阵,故$QBQ^\HH$也是H矩阵,特征是为实数.

$AB$与$QBQ^\HH$具有相同的特征值,故$AB$特征值为实数.

同理,$BA$特征值为实数.
\qed

\newpage
\section{}
设$A$是半正定H矩阵,$A\ne 0$,$B$是正定H矩阵,试证:$|A+B| >|B|$

\answer
由于$B$为正定H矩阵,所以存在可逆矩阵$Q$,使得$B=Q^\HH Q$,于是有
$$|A+B|= |A+Q^\HH Q| = |Q^\HH| |(Q^\HH)^{-1}AQ^{-1}+I| |Q| = |B| |(Q^\HH)^{-1}AQ^{-1}+I|$$

其中矩阵$(Q^\HH)^{-1}AQ^{-1}$仍然是半正定H阵,具有$n$个非负特征值$\lambda_1,\lambda_2,\cdots,\lambda_n$,于是有
$$|(Q^\HH)^{-1}AQ^{-1}+I| > (1+\lambda_1)(1+\lambda_2)\cdots(1+\lambda_n)>1$$

最终有
$$|A+B|= |B| |(Q^\HH)^{-1}AQ^{-1}+I| > |B|$$
\qed

\section{}
\newcommand{\xx}[2]{\overline{x}_{#1}x_{#2}}
已知Hermite二次型
$$f(x_1,x_2,x_3)=\xx{1}{1}+2i\xx{1}{3}-2i\xx{3}{1}+3\xx{2}{2}+\xx{3}{3}$$

求酉变换$X=UY$并将$f(x_1,x_2,x_3)$化为Hermite二次型的标准形.

\answer
二次型矩阵$A=\mat{1&0&2i\\0&3&0\\-2i&0&1}$, $|\lambda I - A|=(\lambda-3)^2(\lambda+1)$

对于特征值$\lambda=3$,解方程组$(3I-A)X=0$可得两个标准单位向量解为
$$\eta_1=[0,1,0]^\TT, \eta_2=[\frac{i}{\sqrt{2}},0,\frac{1}{\sqrt{2}}]^\TT$$

同样地,对于特征值$\lambda=-1$,解得$\eta_3=[\dfrac{1}{\sqrt{2}},0,\dfrac{i}{\sqrt{2}}]^\TT$

$U=[\eta_1,\eta_2,\eta_3]$, $U^\HH AU=\mat{3\\&3\\&&-1}$
\qed


\section{}
设$A,B$为两个正定Hermite矩阵,证明:
$$\mr{det}(A)+\mr{det}(B)\le\mr{det}(A+B)$$
\answer
存在可逆矩阵$P$,使得
$$P^\HH AP=\mat{\lambda_1\\&\ddots\\&&\lambda_n},P^\HH BP=I$$

其中,由于$A$正定, $\lambda_i > 0 (i=1,\cdots,n)$

$|P^\HH AP|= |P^\HH||A||P|=\lambda_1\cdots\lambda_n$

$|P^\HH BP|=|P^\HH||B||P|=1$

再考虑$P^\HH AP + P^\HH BP = P^\HH (A+B)P=\mat{\lambda_1+1\\&\ddots\\&&\lambda_n+1}$,于是有
\begin{align*}
    |P^\HH|| (A+B)||P| & =(\lambda_1+1)\cdots(\lambda_n+1) \\
                       & \ge \lambda_1\cdots\lambda_n + 1  \\
                       & = |P^\HH||A||P| + |P^\HH||B||P|
\end{align*}

又$|P^\HH| \ne 0$,$|P|\ne0$, 最终得$\mr{det}(A)+\mr{det}(B)\le\mr{det}(A+B)$
\qed

\newpage
\section{}
设$\alpha_1=(1,0,i,0)^\TT$,$\alpha_2=(1,1,0,0)^\TT$,求向量$\gamma=(1,1,1,1)^\TT$在子空间$S=\mr{span}\{\alpha_1,\alpha_2\}$上的正交投影.

\answer
将$\alpha_1,\alpha_2$正交标准化为:

$\beta_1=\alpha_1$, $\eta_1=\dfrac{1}{\sqrt{2}}(1,0,i,0)^\TT$

$\beta_2=\alpha_2-\dfrac{(\alpha_2,\beta_1)}{(\beta_1,\beta_1)}\beta_1= \dfrac{1}{2}(1,2,-i,0)^\TT$, $\eta_2=\dfrac{1}{\sqrt{6}}(1,2,-i,0)^\TT$

令$U=[\eta_1,\eta_2]$,投影变换矩阵为$UU^\TT$

向量$\gamma$在$S$的正交投影为$UU^\TT\gamma=(1-\dfrac{i}{3},1+\dfrac{i}{3},\dfrac{2}{3},0)^\TT$
\qed


\section{}
已知四阶矩阵$A=\mat{a&b&c&d\\ b&-a&d&-c\\ c&-d&-a&b\\ d&c&-b&-a}$,其中$a,b,c,d$是全不为零的实数,则线性方程组$AX=\eta$的解为.

\answer
$A^\TT A = AA^\TT = (a^2+b^2+c^2+d^2)I_{4\times 4}$

$X=A^{-1}\eta=(a^2+b^2+c^2+d^2)^{-1}A^\TT\eta$
\qed


\section{}
已知Hermite二次型$f(x_1,x_2,x_3)=3\xx{1}{1}+1\xx{2}{2}+(1+i)\xx{1}{2}+(1-i)\xx{2}{1}+2\xx{3}{3}$,其是( )二次型

\answer
$A=\mat{3&1+i&0\\1-i&1&0\\0&0&2}$,三个顺序主子式均正,正定
\qed


\section{}
已知Hermite二次型$f(x_1,x_2,x_3)=2\xx{1}{1}+i\xx{1}{2}-i\xx{2}{1}+2\xx{2}{2}+2\xx{3}{3}$,其是( )二次型

\answer
$A=\mat{2&i&0\\-i&2&0\\0&0&2}$,三个顺序主子式均正,正定
\qed


\section{}
设矩阵$A$为一个$n$阶正定Hermite且又为酉矩阵,则$A$的特征值为.

\answer
$n$个1
\qed


\section{}
已知$R^3$中向量$\alpha=(1,0,0)^\TT,\beta=(2,0,3)^\TT$,则向量$X=(x_1,x_2,x_3)^\TT\in R^3$在子空间$\mr{span}\{\alpha,\beta\}$上的正交投影为.

\answer
$(x_1,0,x_3)^\TT$
\qed


\end{document}
