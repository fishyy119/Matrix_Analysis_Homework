\documentclass{matrix-homework}

\chapternumber{5}
\begin{document}
\makecover{5}

\section{}
设$A$为$n$阶正定H矩阵,对任意$X\in C^n$,定义$$\norm{X}_A=\sqrt{X^\HH AX}$$试证: $\norm{X}_A$是一种向量范数.

\answer
\textbf{非负性:}\\由于$A$是正定H矩阵,对于任意$X\in C^n$,都有$X^\HH AX\ge0,\norm{X}_A\ge0$\\当且仅当$X=0$时$X^\HH AX=0,\norm{X}_A=0$

\textbf{齐次性:}\\对于任意$k\in C$, $\norm{kX}_A= \displaystyle\sqrt{k\bar{k}X^\HH AX}= |k|\norm{X}_A$

\textbf{三角不等式:}\\由于$A$为正定H矩阵,存在可逆矩阵$B$使得$A=B^\HH B$
\begin{align*}
    \norm{\alpha_1+\alpha_2}^2_A & = (\alpha_1+\alpha_2)^\HH B^\HH B(\alpha_1+\alpha_2)                                                                   \\
                                 & \xlongequal[]{\beta_i=B\alpha_i} (\beta_1+\beta_2)^\HH(\beta_1+\beta_2)                                                \\
                                 & = \beta_1^\HH\beta_1 + \beta_2^\HH\beta_2 + \beta_1^\HH\beta_2 + \beta_2^\HH\beta_1                                    \\
                                 & =\sum_{i=1}^{n}|x_i|^2 + \sum_{i=1}^{n}|y_i|^2 + \sum_{i=1}^{n}\bar{x}_iy_i + \sum_{i=1}^{n}x_i\bar{y}_i               \\
                                 & \xlongequal[]{\bar{z}+z=\mr{Re}(z)}\sum_{i=1}^{n}|x_i|^2 + \sum_{i=1}^{n}|y_i|^2 +2\sum_{i=1}^{n}\mr{Re}(x_i\bar{y}_i) \\
                                 & \le \sum_{i=1}^{n}|x_i|^2 + \sum_{i=1}^{n}|y_i|^2 + 2\sum_{i=1}^{n} |x_iy_i|                                           \\
                                 & \le \sum_{i=1}^{n}|x_i|^2 + \sum_{i=1}^{n}|y_i|^2 + 2\sqrt{\sum_{i=1}^{n}|x_i|^2}\sqrt{\sum_{i=1}^{n}|y_i|^2}          \\
                                 & =\left(\sqrt{\sum_{i=1}^{n}|x_i|^2} + \sqrt{\sum_{i=1}^{n}|y_i|^2}\right)^2                                            \\
                                 & =\left(\sqrt{\beta_1^\HH\beta_1} + \sqrt{\beta_2^\HH\beta_1}\right)^2                                                  \\
                                 & =\left(\sqrt{\alpha_1^\HH A\alpha_1} + \sqrt{\alpha_2^\HH A\alpha_1}\right)^2                                          \\
                                 & =\left(\norm{\alpha_1}_A + \norm{\alpha_2}_A\right)^2
\end{align*}
\qed


\section{}
试证: $\norm{X}=\displaystyle\sqrt{|2x_1-3ix_2+x_3|^2 + |x_2-2x_3|^2 + |x_1+x_3|^2}$是$C^3$上的向量范数.

\answer
$\mat{3 & -3i & 1 \\ 0 & 1 & -2 \\ 1 & 0 & 3}\mat{x_1 \\ x_2 \\ x_3}=AX=\mat{2x_1-3ix_2+x_3 \\ x_2-2x_3 \\ x_1+x_3}$, $\mr{rank}(A)=3$

故$\norm{X}=\norm{AX}_2=\sqrt{X^\HH A^\HH AX}$, 其中$A^\HH A$为正定H矩阵.

(下方证明内容重复题目一证明)

\textbf{非负性:}对于任意$X\in C^n$,都有$\norm{X}=\norm{AX}_2=\sqrt{X^\HH A^\HH AX} \ge0,\norm{X}_2\ge0$, 当且仅当$X=0$时$\norm{X}=0$

\textbf{齐次性:}对于任意$k\in C$, $\norm{kX}=\norm{kAX}_2= \displaystyle\sqrt{k\bar{k}X^\HH A^\HH AX}= |k|\norm{X}$

\textbf{三角不等式:}
\begin{align*}
    \norm{\alpha_1+\alpha_2}^2 & = (\alpha_1+\alpha_2)^\HH A^\HH A(\alpha_1+\alpha_2)                                                                   \\
                               & \xlongequal[]{\beta_i=A\alpha_i} (\beta_1+\beta_2)^\HH(\beta_1+\beta_2)                                                \\
                               & = \beta_1^\HH\beta_1 + \beta_2^\HH\beta_2 + \beta_1^\HH\beta_2 + \beta_2^\HH\beta_1                                    \\
                               & =\sum_{i=1}^{n}|x_i|^2 + \sum_{i=1}^{n}|y_i|^2 + \sum_{i=1}^{n}\bar{x}_iy_i + \sum_{i=1}^{n}x_i\bar{y}_i               \\
                               & \xlongequal[]{\bar{z}+z=\mr{Re}(z)}\sum_{i=1}^{n}|x_i|^2 + \sum_{i=1}^{n}|y_i|^2 +2\sum_{i=1}^{n}\mr{Re}(x_i\bar{y}_i) \\
                               & \le \sum_{i=1}^{n}|x_i|^2 + \sum_{i=1}^{n}|y_i|^2 + 2\sum_{i=1}^{n} |x_iy_i|                                           \\
                               & \le \sum_{i=1}^{n}|x_i|^2 + \sum_{i=1}^{n}|y_i|^2 + 2\sqrt{\sum_{i=1}^{n}|x_i|^2}\sqrt{\sum_{i=1}^{n}|y_i|^2}          \\
                               & =\left(\sqrt{\sum_{i=1}^{n}|x_i|^2} + \sqrt{\sum_{i=1}^{n}|y_i|^2}\right)^2                                            \\
                               & =\left(\sqrt{\beta_1^\HH\beta_1} + \sqrt{\beta_2^\HH\beta_1}\right)^2                                                  \\
                               & =\left(\sqrt{\alpha_1^\HH A^\HH A\alpha_1} + \sqrt{\alpha_2^\HH A^\HH A\alpha_1}\right)^2                              \\
                               & =\left(\norm{\alpha_1} + \norm{\alpha_2}\right)^2
\end{align*}
\qed


\section{}
对于任意的$A\in C^{m\times n}$, 试证: $$\norm{A}=\max\{m,n\}\max_{i,j}|a_{ij}|$$是矩阵范数.

\answer
\textbf{非负性:} $\max\{m,n\}>0, |a_{ij}|\ge0$,故 $\norm{A}\ge0$ 当且仅当$A=0$时$\norm{A}=0$

\textbf{齐次性:} 对于任意$k\in C$, $\norm{kA}= \max\{m,n\}\max_{i,j}|ka_{ij}|= |k|\norm{A}$

\textbf{三角不等式:}
\begin{align*}
    \norm{A+B} & =\max\{m,n\}\max_{i,j}|a_{ij}+b_{ij}|                  \\
               & =   \max\{m,n\}\max_{i,j}(|a_{ij}|+|b_{ij}|)           \\
               & \le \max\{m,n\}(\max_{i,j}|a_{ij}|+\max_{i,j}|b_{ij}|) \\
               & = \norm{A}+\norm{B}
\end{align*}

\textbf{矩阵乘法相容性:}
\begin{align*}
    \norm{AB} & =\max\{m,n\}\max_{i,j}|a_{ij}b_{ij}|                  \\
              & = \max\{m,n\}(\max_{i,j}|a_{ij}||b_{ij}|)             \\
              & \le \max\{m,n\}(\max_{i,j}|a_{ij}|\max_{i,j}|b_{ij}|) \\
              & = \norm{A}\norm{B}
\end{align*}
\qed

\newpage
\section{}
设$$A=\mat{0 & a & a \\ a & 0 & a \\ a & a & 0}$$问$a$取何值时,有$\lim_{k\to\infty}A^k=0$.

\answer
$|\lambda I-A| = (\lambda+a)^2(\lambda-2a)$

$\rho(A)=|2a|$

当$\rho(A)<1$,即$-\frac{1}{2}<a<\frac{1}{2}$时,有$\lim_{k\to\infty}A^k=0$.
\qed


\section{}
判断矩阵幂级数
\begin{align*}
    \sum_{k=0}^{\infty}k\mat{\frac{1}{6} & -\frac{1}{3} \\ -\frac{4}{3} & \frac{1}{6}}^k
\end{align*}
的敛散性.

\answer
收敛半径$R = \dfrac{1}{\lim_{n\to\infty}\left|\frac{n+1}{n}\right|}=1$

矩阵$\mat{\frac{1}{6} & -\frac{1}{3} \\ -\frac{4}{3} & \frac{1}{6}}$特征值为:$\lambda_1=\frac{5}{6},\lambda_2=-\frac{1}{2}$

矩阵谱半径为$\frac{5}{6}<1$,题目中级数绝对收敛
\qed

\newpage
\section{}
已知$u\in C^n(n>1)$为一个单位列向量,令$A=E-uu^\HH$, 试证:\\(1) $\norm{A}_2=1$\\(2) 对任意的$X\in C^n$, 如果有$AX\ne X$,那么$\norm{AX}_2<\norm{X}_2$.

\answer
(1) $A^2=(E-uu^\HH)^2=E-uu^\HH=A$, $A$为幂等矩阵, 其特征值为0或1.

$\mr{rank}(uu^\HH)=1<n$, $|E-A|=|uu^\HH|=0$, 故1是矩阵的一个特征值, $\rho(A)=1$

$A$是H矩阵,也是正规矩阵,其奇异值为特征值模长,故其最大奇异值为1

即$\norm{A}_2=1$

(2) 设$Y=X-AX\ne0$
\begin{align*}
    Y^\HH Y & =(X^\HH-X^\HH A^\HH)(X-AX)                              \\
            & = X^\HH X - X^\HH A^\HH X - X^\HH AX+X^\HH A^\HH AX     \\
            & \xlongequal[]{A^\HH=A=A^\HH A} X^\HH X - X^\HH A^\HH AX
\end{align*}

于是有
\begin{align*}
    \norm{Y}_2=\norm{X}_2 - \norm{AX}_2 \ge 0
\end{align*}
当且仅当$Y=0$,即$AX=X$时取等号.

由题目条件可知等号不会成立,最终有$\norm{AX}_2<\norm{X}_2$.
\qed


\section{}
已知$A=\mat{\lambda & 0 & 0 \\ 1 & \lambda & 0 \\ 0 & 1 & \lambda}$,则$A^{10}$为

\answer
$\mat{\lambda^{10} & 0 & 0 \\ 10\lambda^9 & \lambda^{10} & 0 \\ 45\lambda^8 & 10\lambda^9 & \lambda^{10}}$
\qed


\section{}
$A=\mat{1&1\\0&0\\1&1}$,求谱范数与Frobenius范数

\answer
2,2
\qed


\section{}
$A=\mat{2&-1&0\\ 0&2&4 \\ 1&2&0}$,求1-范数与$\infty$-范数

\answer
5,6
\qed


\section{}
$A=\mat{1 & 0 & 0 \\ 0 & 0.9 & -2 \\ 0 & 0.08 & 0.9}$,求$\lim_{k\to\infty}A^k$

\answer
$A=\mat{1&0\\0&B},A^k=\mat{1&0\\0&B^k}$

令$B=\mat{1&0&0\\0&0&0\\0&0&0}$,$\norm{A^k-B}_1=0$

$\lim_{k\to\infty}A^k=B$
\qed


\section{}
$A=\mat{1/4 & 1/2 \\ 1/2 & 1/4}$,则级数$I+A+A^2+\cdots+A^k+\cdots$为

\answer
$(I-A)^{-1}=\dfrac{4}{5}\mat{3&2\\2&3}$
\qed

\end{document}
