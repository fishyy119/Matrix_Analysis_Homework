\documentclass{matrix-homework}
\usepackage{enumitem}

\newcommand{\OT}[2]{\left(#1\otimes#2\right)}
\newcommand{\inv}[1]{\left(#1\right)^{-1}}

\chapternumber{9}
\begin{document}
\makecover{9}

\section{}
举例说明
\begin{enumerate}[label=(\arabic*)]
    \item $A\otimes B$不一定等于$B\otimes A$
    \item 对于$4\times 4$的矩阵$C$不一定存在$2\times 2$的矩阵$A,B$, 使得$C=A\otimes B$
\end{enumerate}
\answer
\textbf{(1)} $A=\mat{1 & 0 \\ 0 & 1}, B=\mat{0&1\\1&0}$,
$A\otimes B=\mat{B&0\\0&B}\ne B\otimes A=\mat{0&A\\A&0}$

\textbf{(2)} $C=\mat{1&0&0&0\\0&1&0&0\\0&0&0&1\\0&0&1&0}=\mat{I_2 & 0 \\ 0&P}$

设$A=\mat{a&b\\c&d}, A\otimes B=\mat{aB&bB\\cB&dB}$, 条件$\left\{\begin{array}{l}aB=I_2\\dB=P\end{array}\right.$不可能同时满足.
\qed


\section{}
证明:如果$A\in C^{n\times n}, B\in C^{m\times m}$都是正规矩阵,那么$A\otimes B$也是正规矩阵.

\answer
\begin{align*}
    \left(A\otimes B\right)^\HH\left(A\otimes B\right)
     & = \left(A^\HH\otimes B^\HH\right)\left(A\otimes B\right) \\
     & = A^\HH A \otimes B^\HH B                                \\
     & = AA^\HH \otimes BB^\HH                                  \\
     & = \left(A\otimes B\right)\left(A^\HH\otimes B^\HH\right) \\
     & = \left(A\otimes B\right)\left(A\otimes B\right)^\HH
\end{align*}
\qed

\newpage
\section{}
证明:对于任意矩阵$A,B$,都有$\left(A\otimes B\right)^+=A^+\otimes B^+$.

\answer
满秩分解得$A=C_1D_1,B=C_2D_2$, 其中$C_1,C_2$列满秩, $D_1,D_2$行满秩,

易知$C_1\otimes C_2$列满秩, $D_1\otimes D_2$行满秩, $A\otimes B=\left(C_1\otimes C_2\right)\left(D_1\otimes D_2\right)$也是满秩分解.
\newcommand{\A}{\OT{D_1}{D_2}}
\newcommand{\B}{\OT{C_1}{C_2}}
\begin{align*}
     & \OT{A}{B}^+                                                                                                                                     \\
     & = \A^\HH\left(\A\A^\HH\right)^{-1}\left(\B^\HH\B\right)^{-1}\B^\HH                                                                              \\
     & = \A^\HH\OT{D_1D_1^\HH}{D_2D_2^\HH}^{-1}\OT{C_1^\HH C_1}{C_2^\HH C_2}^{-1}\B^\HH                                                                \\
     & = \OT{D_1^\HH}{D_2^\HH}\OT{\left(\inv{D_1D_1^\HH}\inv{C_1^\HH C_1}\right)}{\left(\inv{D_2D_2^\HH}\inv{C_2^\HH C_2}\right)}\OT{C_1^\HH}{C_2^\HH} \\
     & = \left(D_1^\HH\inv{D_1D_1^\HH}\inv{C_1^\HH C_1}C_1^\HH\right) \otimes \left(D_2^\HH\inv{D_2D_2^\HH}\inv{C_2^\HH C_2}C_2^\HH\right)             \\
     & = A^+\otimes B^+
\end{align*}
\qed


\section{}
已知矩阵$$A=\mat{1&0&0\\-1&2&-1\\0&0&2},B=\mat{1&3\\3&1}$$求矩阵$2A\otimes E+A^2\otimes B$的特征值,此处$E$为二阶方阵.

\answer
$A$的特征值为$1,2,2$, $B$的特征值为$-2,4$

令$f(x,y)=2x+x^2y,f(A,B)=2A\otimes E+A^2\otimes B$

$f(A,B)$特征值为$0,-4,-4,6,20,20$
\qed

\newpage
\section{}
证明:如果$A\in C^{n\times n}, B\in C^{m\times m}$, 那么$\norm{A\otimes B}_2=\norm{A}_2\norm{B}_2$.

\answer
\begin{align*}
    \norm{A\otimes B}_2^2
     & = \max_j\lambda_j(\OT{A}{B}^\HH \OT{A}{B})                \\
     & = \max_j\lambda_j\OT{A^\HH A}{B^\HH B}                    \\
     & = \max_{i,j}\lambda_i(A^\HH A)\lambda_j(B^\HH B)          \\
     & = \max_j\lambda_j(A^\HH A) \cdot \max_j\lambda_j(B^\HH B) \\
     & = \norm{A}_2^2\norm{B}_2^2
\end{align*}
所以$\norm{A\otimes B}_2=\norm{A}_2\norm{B}_2$
\qed


\section{}
已知$A=\mat{1&2&3\\3&1&1},B=\mat{2&1\\2&3}$, 那么$A\otimes B=$\_\_\_\_.

\answer
$
    \begin{bmatrix}
        2 & 1 & 4 & 2 & 6 & 3 \\
        2 & 3 & 4 & 6 & 6 & 9 \\
        6 & 3 & 2 & 1 & 2 & 1 \\
        6 & 9 & 2 & 3 & 2 & 3
    \end{bmatrix}
$
\qed


\section{}
已知$A=\mat{2&1&0\\3&4&1\\1&0&2},B=\mat{2&1\\2&3}$, 那么$\mr{Tr}\OT{A}{B}=$\_\_\_\_.

\answer
$\mr{Tr}\OT{A}{B}=\mr{Tr}(A)\mr{Tr}(B)=8\times5=40$
\qed


\section{}
已知$A=\mat{2&1&0\\3&4&1\\-1&2&1},B=\mat{2&1\\2&3}$, 那么$\mr{rank}\OT{A}{B}=$\_\_\_\_.

\answer
$\mr{rank}\OT{A}{B}=(\mr{rank}A)(\mr{rank}B)=4$
\qed


\section{}
已知$A=\mat{1&2&3\\3&1&1}$, 那么$A$的列展开$\mr{cs}(A)=$\_\_\_\_.

\answer
$\mat{1&3&2&1&3&1}^\TT$
\qed


\section{}
已知$A=\mat{1&3&2\\0&2&1\\0&0&2},B=\mat{1&2\\2&1}$, 那么$A^2\otimes B$的特征值为\_\_\_\_.

\answer
$f(x,y)=x^2y$

$-1,-4,-4,3,12,12$
\qed

\end{document}
