\documentclass{matrix-review}
\usepackage{ulem} % 他会把emph变为下划线样式
\usepackage{multicol}

\begin{document}

\section{1}
\subsection{正规矩阵}

\section{向量与矩阵范数}

\subsection{向量范数}
向量范数需要满足三个性质:
\begin{enumerate}
    \item 非负性: $\norm{\al}\ge 0$, 当且仅当$\al=0$时$\norm{\al}=0$
    \item 齐次性: $\norm{k\al}=|k|\norm{\al}$
    \item 三角不等式: $\norm{\al+\be}\le\norm{\al}+\norm{\be}$
\end{enumerate}
常用的三个向量范数: $\norm{\al}_1=\displaystyle\sum_{i=1}^{n}|a_i|;$
$\norm{\al}_2=\displaystyle\sqrt{\sum_{i=1}^{n}|a_i|^2}=\sqrt{\al^\HH\al};$
$\norm{\al}_\infty=\displaystyle\max_{1\le i\le n}|a_i|$

\begin{remarkbox}
    常用不等式1(Holder不等式):\begin{equation*}
        \sum_{i=1}^{n}|a_ib_i|\le\left(\sum_{i=1}^{n}\left|a_i\right|^p\right)^{1/p}\left(\sum_{i=1}^{n}\left|b_i\right|^p\right)^{1/p}
    \end{equation*}其中$p>1,q>1$且$1/p+1/q=1$\\[1em]
    常用不等式2(Minkowski不等式):\begin{equation*}
        \left(\sum_{i=1}^{n}\left|a_i+b_i\right|^p\right)^{1/p}\le\left(\sum_{i=1}^{n}\left|a_i\right|^p\right)^{1/p}+\left(\sum_{i=1}^{n}\left|b_i\right|^p\right)^{1/p}
    \end{equation*}其中$p\ge1$
\end{remarkbox}

向量范数的等价性: 对于两个有限维线性空间$V$上的向量范数$\norm{\al}_a,\norm{\al}_b$, 存在两个与$\al$无关的正数使得\begin{equation*}
    d_1\norm{\al}_b\le \norm{\al}_a\le d_2\norm{\al}_b, \qquad \forall \al\in V
\end{equation*}

利用向量范数构造新范数: 设$\norm{\cdot}_b$是$C^m$上的向量范数,$A\in C^{m\times n}_n$, 则\begin{equation*}
    \norm{\al}_a = \norm{A\al}_b
\end{equation*}所定义的$\norm{\cdot}_a$是$C^n$上的向量范数.

\subsection{矩阵范数}
矩阵范数需要满足四个性质:
\begin{enumerate}
    \item 非负性: $\norm{A}\ge 0$, 当且仅当$A=0$时$\norm{A}=0$
    \item 齐次性: $\norm{kA}=|k|\norm{A}$
    \item 三角不等式: $\norm{A+B}\le\norm{A}+\norm{B}$
    \item \important 矩阵乘法相容性: $\norm{AB}\le\norm{A}\norm{B}$
\end{enumerate}
矩阵范数也有类似于向量范数的等价性,不再赘述.

常见矩阵范数:
\begin{enumerate}
    \item Frobenious范数 $\norm{A}_F^2=\displaystyle\sum_{i=1}^{m}\sum_{i=1}^{n}\left|a_{ij}\right|^2=\Tr{A^\HH A}$
    \item 列和范数 $\norm{A}_1=\displaystyle\max_j\left(\sum_{i=1}^{m}\left|a_{ij}\right|\right)$
    \item 谱范数 $\norm{A}_2=\displaystyle\max_{j}\sqrt{\la_j\left(A^\HH A\right)}$
    \item 行和范数 $\norm{A}_\infty=\displaystyle\max_i\left(\sum_{j=1}^{n}\left|a_{ij}\right|\right)$
\end{enumerate}
\begin{remarkbox}
    Frobenious范数的一些其他性质:
    \begin{enumerate}
        \item $\displaystyle A=\mat{\al_1&\al_2&\cdots&\al_n}, \norm{A}_F^2=\sum_{i=1}^{n}\norm{\al_i}_2^2$
        \item 酉不变性: $\norm{A}_F=\norm{UA}_F=\norm{A^\HH}_F=\norm{AV}_F=\norm{UAV}_F$
        \item Frobenious范数与向量2-范数\emph{相容}
    \end{enumerate}
\end{remarkbox}

\textbf{矩阵范数与向量范数相容}: 对于任意矩阵$A$与向量$X$都有\begin{equation*}
    \norm{AX}_\al\le\norm{A}_\be\norm{X}_\al
\end{equation*}则称矩阵范数$\norm{A}_\be$与向量范数$\norm{X}_\al$是相容的.

\textbf{诱导范数(算子范数)}: 设$\norm{X}_\al$是向量范数,则\begin{equation*}
    \norm{A}_i=\max_{X\ne0}\frac{\norm{AX}_\al}{\norm{X}_\al}
\end{equation*}是与$\norm{X}_\al$ \emph{相容}的矩阵范数,被称为算子范数.

\textbf{由矩阵范数构造向量范数}: 设$\norm{A}_*$是矩阵范数,则存在与其\emph{相容}的向量范数$\norm{X}=\norm{X\al^\HH}_*$.

\textbf{\important 谱半径}: 矩阵$A$所有特征值的绝对值中的最大值被称为矩阵$A$的谱半径,即\begin{equation*}
    \rho(A)=\max\left\{|\la_1|,|\la_2|,\cdots,|\la_n|\right\}
\end{equation*}矩阵$A$的任意一种范数都不小于它的谱半径,即$\rho(A)\le\norm{A}$.

\subsection{矩阵序列与极限}
矩阵各项收敛即可得到矩阵序列收敛, 除加法数乘外还有如下运算规则:
\begin{enumerate}
    \item $\displaystyle\lim_{k\to\infty}A^{(k)}B^{(k)}=AB$
    \item $\displaystyle\lim_{k\to\infty}PA^{(k)}Q=PAQ$
    \item $\displaystyle\lim_{k\to\infty}\left(A^{(k)}\right)^{-1}=A^{-1}$
\end{enumerate}

矩阵序列$\left\{A^{(k)}\right\}$收敛于$A$的充要条件为$\displaystyle\lim_{k\to\infty}\norm{A^{(k)}-A}=0$.
\begin{remarkbox}
    根据矩阵范数等价性,任意范数成立上式后都可以使用夹逼准则证明另一范数满足该条件,因此证明时可以使用最简单形式的范数$\norm{A}=\displaystyle\sum_{i=1}^{m}\sum_{j=1}^{n}|a_{ij}|$.
\end{remarkbox}

矩阵序列$\left\{A,A^2,\cdots,A^k,\cdots\right\}$收敛于$0$的充要条件为$\rho(A)<0$.

\subsection{矩阵幂级数}
矩阵级数$\displaystyle\sum_{k=1}^{\infty}A^{(k)}$绝对收敛的充要条件是正项级数$\displaystyle\sum_{k=1}^{\infty}\norm{A^{(k)}}$收敛.

矩阵幂级数$\displaystyle\sum_{k=0}^{\infty}a_kA^k$在$\rho(A)<R$时绝对收敛,在$\rho(A)>R$时发散,在$\rho(A)=R$时需额外判断.其中$R$为收敛半径,计算公式如下\begin{equation*}
    \frac{1}{R}=\lim_{k\to\infty}\left|\frac{a_{k+1}}{a_k}\right|
\end{equation*}
\begin{remarkbox}
    因为谱半径是所有范数值的下界,因此可以通过寻找在收敛域内的范数值来说明级数收敛.
\end{remarkbox}

\textbf{\important Jordan块矩阵的幂}: \begin{equation}
    J_i^k(\la_i)=\begin{bmatrix}
        \la_i^k & C^1_k\la_i^{k-1} & \cdots & C_k^{d_i-1}\la_i^{k-d_i+1} \\
                & \la_i^k          & \ddots & \vdots                     \\
                &                  & \ddots & C^1_k\la_i^{k-1}           \\
                &                  &        & \la_i^k
    \end{bmatrix}_{d_i\times d_i}
\end{equation}
\begin{remarkbox}
    常见函数的幂级数展开, 前三个全域收敛, 后三个收敛半径为$1$
    \begin{multicols}{2}
        \begin{enumerate}
            \item $e^x=\displaystyle\sum_{n=0}^{\infty}\frac{x^n}{n!}$
            \item $\sin x=\displaystyle\sum_{n=0}^{\infty}\frac{(-1)^n}{(2n+1)!}x^{2n+1}$
            \item $\cos x=\displaystyle\sum_{n=0}^{\infty}\frac{(-1)^n}{(2n)!}x^{2n}$
            \item $\arctan x=\displaystyle\sum_{n=1}^{\infty}\frac{(-1)^{n-1}}{2n-1}x^{2n-1}$
            \item $\ln x=\displaystyle\sum_{n=1}^{\infty}\frac{(-1)^{n+1}}{n}x^{n}$
            \item $\displaystyle\frac{1}{1+x}=(1+x)^{-1}=\sum_{i=0}^{\infty}(-1)^nx^n$
        \end{enumerate}
    \end{multicols}
\end{remarkbox}

一个特殊级数:矩阵幂级数\begin{equation*}
    I+A+A^2+\cdots+A^k+\cdots
\end{equation*}绝对收敛的充要条件是$\rho(A)<1$,收敛和为$(I-A)^{-1}$.
\begin{remarkbox}
    充分性证明是前文定理,必要性证明是$\lim_{k\to\infty}\norm{A^k}=0$.
\end{remarkbox}


\section{矩阵函数}
\newcommand{\ddd}[2]{\dfrac{\dd #1}{\dd #2}}
\newcommand{\ddx}[1]{\ddd{#1}{x}}
\newcommand{\ddt}[1]{\ddd{#1}{t}}

\subsection{矩阵多项式}
$\displaystyle p(A)=\sum_{i=0}^{m}a_iA^i$被称为矩阵多项式. 考虑$d_i$阶Jordan块矩阵$J_i$,则\begin{equation}
    p(J_i)=\begin{bmatrix}
        p(\la_i) & p'(\la_i) & \frac{p''(\la_i)}{2!} & \cdots & \frac{p^{(d_i-1)}(\la_i)}{(d_i-1)!} \\
                 & p(\la_i)  & p'(\la_i)             & \ddots & \vdots                              \\
                 &           & p(\la_i)              & \ddots & \frac{p''(\la_i)}{2!}               \\
                 &           &                       & \ddots & p'(\la_i)                           \\
                 &           &                       &        & p(\la_i)
    \end{bmatrix}_{d_i\times d_i}
\end{equation}
\begin{remarkbox}
    注意其中的阶乘
\end{remarkbox}

\textbf{\important 矩阵多项式的Jordan表示:} $p(A)=P\mr{diag}(p(J_1),p(J_2),\cdots,p(J_r))P^{-1}$, 其中$P$是将$A$变换为Jordan标准形的变换矩阵.
\begin{remarkbox}
    矩阵多项式$p(A)$的特征值对应为$p(\la_i)$,对应的特征向量不变. (使用定义证明)
\end{remarkbox}

\textbf{化零多项式:} 满足$f(A)=O_{n\times n}$的多项式,被称为矩阵$A$的化零多项式. 根据Hamilton-Cayley定理, \emph{矩阵$A$的特征多项式$\mr{det}(\la E-A)$是化零多项式}.

\textbf{最小多项式:} 次数最低且首项系数为1的化零多项式为最小多项式,记为$m(\la)$. 性质如下:
\begin{enumerate}
    \item 任何一个化零多项式都能被$m(\la)$整除
    \item 最小多项式唯一
    \item 相似矩阵具有相同的最小多项式
\end{enumerate}
最小多项式的求法如下:
\begin{enumerate}
    \item 化为Jordan标准形,相似矩阵最小多项式相同,只需考虑标准形
    \item 对于$d_i$阶Jordan块矩阵,简单运算一下即可得到其最小多项式为$(\la-\la_i)^{d_i}$
    \item 准对角矩阵的最小多项式为各块最小多项式的最大公倍式
\end{enumerate}

\subsection{矩阵函数}
\textbf{定义:} $A$的最小多项式为\begin{equation*}
    m(\la)=\sum_{i=1}^{s}(\la-la_i)^{d_ii}  \qquad  \left(\sum_{i=1}^{s}d_i=m\right)
\end{equation*}如果函数$f(x)$具有足够高阶的导数并且以下$m$个值
$$\{f(\la_i),f'(\la_i),\cdots,f^{(d_i-1)}(\la_i),i=1,2,\cdots,s\}$$存在,则称\emph{$f(x)$在矩阵$A$的影谱上有定义}.

$f(x)$在影谱上有定义,如果存在多项式$p(\la)$的上面$m$个值与$f(\la)$相等,则定义矩阵函数$f(A)=p(A)$. 满足定义的多项式\emph{存在但并不唯一}, 只要影谱上对应的值相等, 最终矩阵函数就相等.

\textbf{\important 矩阵函数的Jordan表示:} 与上文矩阵多项式的Jordan表示是相似的,不过是把多项式换为了函数,同时要求在影谱上有定义.

\textbf{\important 矩阵函数的多项式表示:} 使用拉格朗日插入计算多项式系数$a_0,a_1,\cdots,a_{m-1}$,使得多项式影谱上对应值相等\begin{equation*}
    p^{(k)}(\la_i)=f^{(k)}(\la_i), \quad i=1,2,\cdots,s; \quad k=1,2,\cdots,d_i-1
\end{equation*}最终可得到多项式表示$f(A)=a_{m-1}A^{m-1}+\cdots+a_1A+A_0I$

\textbf{\important 矩阵函数的幂级数表示:} 将一元函数$f(x)$展开为收敛半径$R$的幂级数$\displaystyle\sum_{k=0}^{\infty}c_kx^k$. 当矩阵$A$的谱半径$\rho(A)<R$时,矩阵幂级数$\displaystyle\sum_{k=0}^{\infty}c_kx^k$绝对收敛, 该级数即为急诊的幂级数表示.

\subsection{矩阵指数函数和矩阵三角函数}

\begin{align}
    e^{At}  & = \sum_{k=0}^{\infty}\frac{1}{k!}A^Kt^k                     \\
    \sin At & = \sum_{k=0}^{\infty}\frac{(-1)^k}{(2k+1)!}A^{2k+1}t^{2k+1} \\
    \cos At & = \sum_{k=0}^{\infty}\frac{(-1)^k}{(2k)!}A^{2k}t^{2k}
\end{align}
\begin{remarkbox}
    复指数函数的一些结论: $e^{iA}=\cos A+i\sin A$, $e^{-iA}=\cos A-i\sin A$, \\[0.5em]
    $\cos A=\dfrac{e^{iA}+e^{-iA}}{2}$, $\sin A=\dfrac{e^{iA}-e^{-iA}}{2i}$
\end{remarkbox}

当$AB=BA$时,一些标量三角函数的结论可套用:
\begin{enumerate}
    \item $e^{A+B}=e^Ae^B=e^Be^A$ (若$AB$不可交换,该性质不一定成立,但也可能成立)
    \item 三角函数两角和, 二倍角, 平方和
\end{enumerate}

几个特殊性质如下:
\begin{enumerate}
    \item $\ddt{e^{At}}=Ae^{At}=e^{At}A$
    \item $\ddt{\sin At}=A(\cos At)=(\cos At)A$
    \item $\ddt{\cos At}=-A(\sin At)=-(\sin At)A$
    \item $|e^{At}|=e^{\Tr{A}}$
\end{enumerate}

\section{函数矩阵与矩阵微分方程}

\subsection{函数矩阵的导数与积分}
本部分并没有什么值得一提的东西,基本都是把多个标量函数摞在一起运算了一下. 需要注意一些由于矩阵乘法导致交换律不适用的运算:
\begin{enumerate}
    \item $\ddx{}\left[A(x)B(x)\right]=\ddx{A}B+A\ddx{B}$
    \item $\ddx{A^{-1}(x)}=-A^{-1}\ddx{A}A^{-1}$
\end{enumerate}
以及一些做题时经常用到的公式:
\begin{equation}
    \ddx{}\left(\int_{a(x)}^{b(x)}f(t)\dd t\right) = f(b(x))b'(x) - f(a(x))a'(x)
\end{equation}

\subsection{矩阵微分方程}
\sout{我看往年题都不考这里,先暂缓整理吧.}

\section{广义逆与最小二乘}

\subsection{广义逆}
广义逆矩阵$A^-$的充要条件为\begin{equation}\label{eq:inv1}
    AA^-A=A
\end{equation}具有如下性质:
\begin{enumerate}
    \item $A^-b$是方程组$Ax=b$的解 \textbf{(定义)}
    \item $(A^-)^\TT=(A^\TT)^-$
    \item $\Rank{A}\le\Rank{A^-}$
    \item $AA^-$和$A^-A$都是幂等矩阵
\end{enumerate}

\begin{remarkbox}
    根据定义就可以看出,广义逆是不一定唯一的
\end{remarkbox}

\textbf{定义:} 若$A_L^{-1}A=E_n$,则$A_L^{-1}$被称为左逆; 若$AA^{-1}_R=E_m$,则$A_R^{-1}$被称为右逆.
$A$列满秩的充要条件是$A^-A=E$, $A$行满秩的充要条件是$AA^-=E$.

\subsection{自反广义逆}
在\eqref{eq:inv1}的基础上附加另一条件\begin{equation}\label{eq:inv2}
    A^-AA^-=A^-
\end{equation}
满足该定义的广义逆成为自反广义逆,记作$A_r^-$,相比于广义逆具有如下额外性质:
\begin{enumerate}
    \item $\Rank{A}=\Rank{A^-}$ \textbf{(充要条件)}
    \item 若$X,Y$是$A$的广义逆,则$XAY$是自反广义逆
    \item 若$A=BC$为满秩分解,则$C^{-1}_RB^{-1}_L$是自反广义逆
\end{enumerate}

\subsection{伪逆}
在\eqref{eq:inv1}\eqref{eq:inv2}的基础上额外添加两个对称条件,构成Penros-Moore方程如下:\begin{equation}
    \left\{
    \begin{aligned}
        AA^+A                 & =A    \\ A^+AA^+&=A^+                \\
        \left(AA^+\right)^\HH & =AA^+ \\ \left(A^+A\right)^\HH&=A^+A
    \end{aligned}
    \right.
\end{equation}
满足上述四个条件的特殊自反广义逆矩阵即为\textbf{伪逆矩阵},具有如下性质:
\begin{enumerate}
    \item 若$A=BC$是满秩分解,则$C^\HH\left(CC^\HH\right)^{-1}\left(B^\HH B\right)^{-1}B^\HH$是伪逆矩阵 \important
    \item $A^+=U\Lambda^+U^\HH A^\HH$, 其中$U^\HH A^\HH AU=\mr{diag}\{\la_1,\la_2,\cdots,\la_n\}=\Lambda$
    \item $\left(A^+\right)^+=A$
    \item $\left(AA^\HH\right)^+=\left(A^\HH\right)^+A^+=\left(A^+\right)^\HH A^+$, $\left(A^\HH A\right)^+=A^+\left(A^\HH\right)^+=A^+\left(A^+\right)^\HH $
    \item $A^+=A^\HH\left(AA^\HH\right)^+=\left(A^\HH A\right)^+A^\HH$
\end{enumerate}
\begin{remarkbox}
    伪逆是唯一的. (思路:$X=XAX=...=XAY=...=YAY=Y$)
\end{remarkbox}

\subsection{线性方程组}
矩阵方程$AXB=D$有解的充要条件为存在广义逆$A^-,B^-$使得\begin{equation}
    AA^-DB^-B=D
\end{equation}成立. 矩阵方程的通解为\begin{equation}
    X=A^-DB^-+Y-A^-AYBB^-
\end{equation}其中$Y$为任意符合尺寸的矩阵.

\textbf{推论1:} 矩阵方程$AX=D$有解的充要条件为存在广义逆$A^-$使得\begin{equation}
    AA^-D=D
\end{equation}成立. 矩阵方程的通解为\begin{equation}
    X=A^-D+Y-A^-AY
\end{equation}其中$Y$为任意符合尺寸的矩阵.

\textbf{推论2\important :} 矩阵方程$AX=b$有解的充要条件为存在广义逆$A^-$使得\begin{equation}
    AA^-b=b
\end{equation}成立. 矩阵方程的通解为\begin{equation}
    X=A^-b+(E_n-A^-A)Y
\end{equation}其中$Y$为任意符合尺寸的向量.

\textbf{定义:} 相容(有解)方程组$Ax=b$的所有解中模(2-范数)最小的解称为最小模解.

\textbf{定义:} 若$n$维向量$x_0$满足对于任何一个$n$维向量$x$,都有$\norm{Ax_0-b}^2\le\norm{Ax-b}^2$,则称$x_0$是方程组$Ax=b$的一个最小二乘解.

\textbf{定义:} 若对于任意一个最小二乘解$x_0$,最小二乘解$u$都满足$\norm{u}\le\norm{x_0}$,则称$u$为最佳最小二乘解.

最小二乘解有如下性质:
\begin{enumerate}
    \item 对于满足P-M方程\{1,3\}的广义逆矩阵$B$,$x=Bb$是$Ax=b$的最小二乘解
    \item $x=A^+b$是$Ax=b$的最佳最小二乘解 \important
    \item $\norm{Ax^*-b}_2$是$b$到$R(A)$的最短距离,其中$x^*$为最佳最小二乘解 \important
\end{enumerate}

\section{Kronecker积}
\subsection{基本性质}
\newcommand{\OT}[2]{#1 \otimes #2}
\newcommand{\OTL}[2]{\left(#1\right) \otimes #2}
\newcommand{\OTR}[2]{#1 \otimes \left(#2\right)}
\newcommand{\OTLR}[2]{\left(#1\right) \otimes \left(#2\right)}

$\OT{A}{B}=\mat{a_{11}B & a_{12}B & \cdots & a_{1n}B \\
        a_{21}B & a_{22}B & \cdots & a_{2n}B \\
        \vdots & \vdots & \ddots & \vdots \\
        a_{m1}B & a_{m2}B & \cdots & a_{mn}B \\}$, 有如下性质:
\begin{enumerate}
    \item 无交换律. 有结合律、分配律,数乘可任意移动位置
    \item $\left(\OT{A}{B}\right)\left(\OT{C}{D}\right) = \OT{AC}{BD}$  \important
    \item $\left(\OT{A}{B}\right)^\HH = \OT{A^\HH}{B^\HH}$
    \item $\left(\OT{A}{B}\right)^{-1} = \OT{A^{-1}}{B^{-1}}$
    \item $\Tr{\OT{A}{B}} = \Tr{A}\cdot\Tr{B}$
    \item $\Rank{\OT{A}{B}} = \Rank{A}\cdot\Rank{B}$
    \item $|\OT{A}{B}| = {|A|}^p{|B|}^m$ ($p,m$为阶数)
    \item 若$A,B$为正规矩阵、对称矩阵、H矩阵、反H矩阵、酉矩阵, $\OT{A}{B}$保持对应性质
    \item 存在合同变换,使得$P^\TT\left(\OT{A}{B}\right)P=\OT{B}{A}$, 其中$P$为有限个初等矩阵乘积
    \item $\OT{A}{B}\sim\OT{B}{A}$ (上面一条的$P$是正交矩阵)
    \item Kronecker积同样有幂的概念,$\left(AB\right)^{[k]}=A^{[k]}B^{[k]}$
\end{enumerate}

\subsection{特征值}
考虑$\displaystyle f(x,y)=\sum_{i,j}c_{ij}x^iy^j$,
则$\displaystyle f(A,B)=\sum_{i,j}c_{ij}A^i\otimes B^j$的特征值为$f(\la_r,\mu_s)$, 对应特征向量为 $\OT{x_r}{y_s}$,
(其中$\la_i,x_i$为$A$的特征值与特征向量, $\mu_j,y_j$为$B$的...)

\textbf{推论}: $\OT{A}{E_n}+\OT{E_m}{B}$被称为Kronecker和,特征值为$\la_r+\mu_s$

\subsection{行展开与列展开}
\newcommand{\rs}[1]{\mr{rs}\left(#1\right)}
\newcommand{\cs}[1]{\mr{cs}\left(#1\right)}
将矩阵$A$的每一行串接为一个行向量,成为$A$的行展开,记为$\mr{rs}(A)$.
同理,将矩阵$A$的每一列串接为一个列向量,成为$A$的列展开,记为$\mr{cs}(A)$,有如下性质:
\begin{enumerate}
    \item $\rs{A^\TT}=\left(\cs{A}\right)^\TT$,  $\cs{A^\TT}=\left(\rs{A}\right)^\TT$
    \item $\rs{ABC}=\rs{B}\left(\OT{A^\TT}{C}\right)$
    \item $\cs{ABC}=\left(\OT{C^\TT}{A}\right)\cs{B}$
    \item $\cs{AX}=\left(\OT{E_n}{A}\right)\cs{X}$
    \item $\cs{XB}=\left(\OT{B^\TT}{E_m}\right)\cs{X}$
\end{enumerate}

\end{document}