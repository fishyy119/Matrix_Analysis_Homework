\documentclass{matrix-review}
\usepackage{ulem} % 他会把emph变为下划线样式
\usepackage{multicol}

% 孤行寡行
\widowpenalty=10000
\clubpenalty=10000

\begin{document}
\makecover


\section{线性空间与线性变换}

\subsection{线性空间}
\begin{multicols}{2}[\textbf{线性空间定义}: 已知数域$F$,在集合$V$中定义数乘与加法运算,线性空间应满足如下八条运算律:]
    \begin{enumerate}
        \item 加法交换律
        \item 加法结合律
        \item 存在零元素,使得$\forall \al\in V,\al+0=\al$
        \item $\forall \al\in V, \exists \be\in V, \text{s.t. } \al+\be=0$
        \item $1\cdot\al=\al$
        \item $k(l\al)=(kl)\al$
        \item $(k+l)\al=k\al+l\al$
        \item $k(\al+\be)=k\al+k\be$
    \end{enumerate}
\end{multicols}

线性无关,线性相关,极大线性无关组,向量组的秩,基底,坐标,维数...

\textbf{基变换与坐标变换}: 设旧基底$\al_1,\al_2,\cdots,\al_n$与新基底$\be_1,\be_2,\cdots,\be_n$之间变换关系为\begin{equation*}
    \be_i=\mat{\al_1&\al_2&\cdots&\al_n}\mat{a_{1i}\\a_{2i}\\\vdots\\a_{ni}}
\end{equation*}则$P=\mat{a_{11}&a_{12}&\cdots&a_{1n}\\a_{21}&a_{22}&\cdots&a_{2n}\\\cdots&\cdots&\cdots&\cdots\\a_{n1}&a_{n2}&\cdots&a_{nn}}$是由旧基底到新基底的过渡矩阵,基变换公式为:\begin{equation}
    \mat{\be_1&\be_2&\cdots&\be_n}=\mat{\al_1&\al_2&\cdots&\al_n}P
\end{equation}对于同一个向量,其在两个基底下的坐标(y新x旧)变换公式为:\begin{equation}\label{eq:guoduP}
    \mat{x_1\\x_2\\\vdots\\x_n}=P\mat{y_1\\y_2\\\vdots\\y_n}
\end{equation}

\subsection{线性子空间}
设$V$是数域$F$上一个线性空间,$W$是$V$的一个非空子集合,如果$\forall \al,\be\in W,\forall k,l\in F$都有$$k\al+l\be\in W$$则称$W$是$V$的一个子空间.

两个子空间$V_1,V_2$的交空间$V_1\cap V_2$与和空间$V_1+V_2$仍然是子空间. 如果$V_1\cap V_2={0}$,则称和空间为直和,记为$V_1\oplus V_2$.

\textbf{\important 维数公式}:$\Dim{V_1+V_2}=\Dim{V_1}+\Dim{V_2}-\Dim{V_1\cap V_2}$

\subsection{线性映射}
设$\phi$是$V_1$到$V_2$的一个线性映射,令$$\phi(V_1)=\{\be = \phi(\al)\in V_2 | \forall \al \in V_1\}$$则有下列定义:
\begin{enumerate}
    \item $\phi$的值域$R(\phi)$为$\phi(V_1)$
    \item $\phi$的维数$\Rank{\phi}$为$\Dim{R(\phi)}$
    \item $\phi$的核子空间为$N(\phi)=\phi^{-1}(0)=\left\{\al\in V_1 | \phi(\al)=0\right\}$
    \item $\phi$的零度为$\Dim{N(\phi)}$
\end{enumerate}
\begin{remarkbox}
    若零度为0,则线性无关组$\al_1,\al_2,\dots,\al_r\in V_1$的像$\phi(\al_1),\phi(\al_2),\dots,\phi(\al_r)\in V_2$也线性无关
\end{remarkbox}

秩与零度定理: 若$\phi$是$n$维线性空间$V_1$到$m$维线性空间$V_2$的线性映射, 则有$$\Dim{R(\phi)}+\Dim{N(\phi)}=\Dim{V_1}=n$$

线性映射的矩阵表示: 设$\phi\in\mathcal{L}(V_1,V_2)$, 在一组特定的基下表示两个空间的元素为$\al\in V_1,\be\in V_2$, 则存在矩阵表示$A$使得$\be=A\al$.
\begin{remarkbox}
    同一线性变换在不同基下有不同表示,这些不同的矩阵表示相似,即$B=P^{-1}AP$,其中$B$为在新基下的表示,$P$是过渡矩阵\eqref{eq:guoduP}

    显然这就是相当于在$A$的输入输出处进行了接管,转换了一下基
    \begin{remarkbox}
        线性变换是线性映射的特例,映射回同一个空间,即$\mathcal{L}(V,V)$
    \end{remarkbox}
\end{remarkbox}


\subsection{\important 特征值与特征向量}\label{subsec:pq}
特征值与特征向量的定义: $AX=\la X$.

特征值的性质:
\begin{multicols}{2}
    \begin{enumerate}
        \item $|A| =\prod_{i=1}^{n}\la_i$
        \item $\Tr{A} = \sum_{i=1}^{n}\la_i$
    \end{enumerate}
\end{multicols}


已知矩阵$A$的特征多项式为
$$|\la I-A|=(\la-\la_1)^{p_i}(\la-\la_2)^{p_2}\cdots(\la-\la_r)^{p_r}$$
则$p_i$为$\la_i$的\emph{代数重复度}.

属于$\la_i$的全部特征向量加上零向量构成了特征值$\la_i$的\emph{特征子空间}, 它是$\la_iI-A$的零空间.特征子空间的维数$q_i$称为$\la_i$的\emph{几何重复度},显然
$$q_i=n-\Rank{\la_iI-A}$$

\subsection{相似对角形}
$A$可相似对角化的充要条件:
\begin{enumerate}
    \item 有$n$个线性无关的特征向量
    \item 每个特征值的几何重复度等于代数重复度
\end{enumerate}

$A,B$可以同时相似对角化的充要条件为:$AB=BA$


\section{\texorpdfstring{$\la-$矩阵与Jordan标准形}{lambda-矩阵与Jordan标准形}}

\subsection{\texorpdfstring{$\la-$矩阵}{lambda-矩阵}}
称呼每个元素为多项式的矩阵为多项式矩阵($\la$-矩阵),其中各元素最高的次数称为矩阵的次数,各项规律基本与数字矩阵相似. 经过初等行列变换后得到的矩阵之间相互等价.
\begin{remarkbox}
    此处行列变换只能进行对换、单行数乘、倍加三种运算
\end{remarkbox}

\textbf{\important Smith标准形}: 任意一个非零$\la$-矩阵都能等价于对角矩阵\begin{equation*}
    \begin{bmatrix}
        d_1(\la)                                         \\
         & d_2(\la)                                      \\
         &          & \ddots                             \\
         &          &        & d_r(\la)                  \\
         &          &        &          & 0              \\
         &          &        &          &   & \ddots     \\
         &          &        &          &   &        & 0
    \end{bmatrix}
\end{equation*}其中$d_i(\la)$能被$d_{i+1}(\la)$整除, 其中$d_i(\la)$被称为\emph{不变因子}.

\textbf{行列式因子}: 对于所有非0的$k$阶子式,令其首项系数为1后求出的最大公因式$D_k(\la)$被称为$A(\la)$的$k$阶\emph{行列式因子}. 行列式因子的数量等于$\mr{rank}(A(\la))$.
\begin{remarkbox}
    注意这里子式是可以跨行选取的.\\
    等价的矩阵具有相同的不变因子与行列式因子.
\end{remarkbox}

\textbf{\important 行列式因子与不变因子的关系}: $d_1(\la)=D_1(\la),d_2(\la)=\frac{D_2(\la)}{D_1(\la)},\cdots,d_r(\la)=\frac{D_r(\la)}{D_{r-1}(\la)}$. 从而可证明Smith标准形唯一.

\textbf{\important 初等因子}: 由于不变因子的整除特性, 不变因子可以被分解为一次因式幂的乘积:\begin{align*}
    d_1(\la) & =(\la-\la+_1)^{e_{11}}(\la-\la+_2)^{e_{12}}\cdots(\la-\la+_s)^{e_{1s}} \\
    d_2(\la) & =(\la-\la+_1)^{e_{21}}(\la-\la+_2)^{e_{22}}\cdots(\la-\la+_s)^{e_{2s}} \\
    \vdots                                                                            \\
    d_r(\la) & =(\la-\la+_1)^{e_{r1}}(\la-\la+_2)^{e_{r2}}\cdots(\la-\la+_s)^{e_{rs}} \\
\end{align*}其中所有指数大于0的因子$(\la-\la_j)^{e_{ij}}$即为\emph{初等因子},在列出时就算相同也要写多次.
\begin{remarkbox}
    两个$\la$矩阵等价的充要条件是具有相同的不变(行列式/初等)因子.\\
    两个数字矩阵相似的充要条件是他们的特征矩阵$\la I-A$与$\la I-B$等价.
\end{remarkbox}

\subsection{Jordan标准形}
对于一个矩阵$A$,其特征矩阵$\la I-A$的每个初等因子都对应了一个Jordan块矩阵. 例如$(\la-a_i)^{n_i}$对应Jordan块\begin{equation*}
    \begin{bmatrix}
        a_1 & 1                           \\
            & a_1 & 1                     \\
            &     & \ddots & \ddots       \\
            &     &        & \ddots & 1   \\
            &     &        &        & a_i
    \end{bmatrix}_{n_i\times n_i}
\end{equation*}该Jordan块矩阵只有一个初等因子,就是他对应的那个初等因子.于是矩阵就可以相似变换为Jordan标准形\begin{equation}
    P^{-1}AP=
    \begin{bmatrix}
        J_1                   \\
         & J_2                \\
         &     & \ddots       \\
         &     &        & J_s
    \end{bmatrix}
\end{equation}
\begin{remarkbox}
    前面$\la-$矩阵的练习题初等因子都很多,这是因为那些题里的给的矩阵次数都很高,而求Jordan标准形时考虑的是特征矩阵$\la I-A$.
\end{remarkbox}

\textbf{\important 相似变换矩阵的求法}: 将$P$按列分块为$P=\mat{X_1&X_2&\cdots&X_n}$, 获得联立方程组$AP=PJ$, 逐步解方程获得一组解.
\begin{remarkbox}
    这里面涉及到齐次方程与非齐次方程的求解.\\
    齐次方程通解是基础解系的加权和,有时需要选取以保证后面方程有解.\\
    非齐次方程解的形式是齐次方程通解加一个特解.
\end{remarkbox}

\subsection{\important 另一种Jordan标准形的求法}
给定特征值$\la_i$,其对应的Jordan块有如下性质(重复度计算方法见\ref{subsec:pq})
\begin{enumerate}
    \item 对应的Jordan块个数等于\emph{几何重复度}
    \item 对应的Jordan块阶数和等于\emph{代数重复度}
\end{enumerate}

对于阶数很高的情况,分析方法如下:
\begin{enumerate}
    \item 计算$\Rank{\la_iI-A},\Rank{\la_iI-A}^2\cdots$
    \item 对应$\la_i$的Jordan块总数为$n-\Rank{\la_iI-A}$
    \item 对应$\la_i$的Jordan块中大于2阶的数量为$\Rank{\la_iI-A}-\Rank{\la_iI-A}^2$
    \item 大于3阶的有...依此类推,直到不再降秩
\end{enumerate}
\begin{remarkbox}
    考虑$\la_iI-J$的多次幂,不难发现只有对应$\la_i$的Jordan块会降秩,同时在次数为$r$时,阶数小于等于$r$的Jordan块会降秩到0.\\
    上面步骤中不再降秩,就代表该特征值对应的所有Jordan块都已经降秩到0了.
\end{remarkbox}


\section{内积空间与正规矩阵}
\newcommand{\cross}[2]{\left(#1,#2\right)}
\newcommand{\schmidt}[4]{\frac{\cross{#1}{#2}}{\cross{#3}{#4}}}

\subsection{内积空间}
实数域内积的性质:
\begin{multicols}{2}
    \begin{enumerate}
        \item $\cross{\al}{\be}=\cross{\be}{\al}$
        \item $\cross{k\al}{\be}=k\cross{\al}{\be}$
        \item $\cross{\al+\be}{\gamma}=\cross{\al}{\gamma}+\cross{\be}{\gamma}$
        \item $\cross{\al}{\al}\ge0$, 当且仅当$\al=0$时取零
    \end{enumerate}
\end{multicols}

复数域内积的性质:
\begin{multicols}{2}
    \begin{enumerate}
        \item $\cross{\al}{\be}=\overline{\cross{\be}{\al}} $
        \item ...其余三条相同
    \end{enumerate}
\end{multicols}

定义了内积的$n$维实线性空间被称为\emph{欧氏空间},复线性空间被称为\emph{酉空间},两者合称\emph{内积空间}.

\textbf{\important Schmidt正交化方法}: 从一个线性无关向量组$\left\{\al_1,\al_2,\cdots,\al_r\right\}$出发,可以构造一个等价的标准正交向量组(是$\mr{span}\left\{\al_1,\al_2,\cdots,\al_r\right\}$的一个标准正交基):\begin{equation}\label{eq:schmidt}
    \left\{
    \begin{aligned}
        \be_1  & =\al_2                                                                                                              \\
        \be_2  & =\al_2-\schmidt{\al_2}{\be_1}{\be_1}{\be_1}\be_1                                                                    \\
        \cdots & \cdots\cdots\cdots\cdots                                                                                            \\
        \be_r  & =\al_r-\schmidt{\al_2}{\be_1}{\be_1}{\be_1}\be_1 -\cdots- \schmidt{\al_r}{\be_{r-1}}{\be_{r-1}}{\be_{r-1}}\be_{r-1}
    \end{aligned}
    \right.
\end{equation}标准化步骤不再赘述.

\textbf{正交矩阵(酉矩阵)}: 转置(共轭转置)等于逆的矩阵, 其各行(列)向量构成标准正交向量组, 行列式模长为1. \sout{(可以从几何直观上理解,懒得写了,反正我知道...)}

\subsection{幂等矩阵}
满足式$$A^2=A$$的矩阵被称为幂等矩阵,其\emph{充要条件}是可以被相似对角化为$\mat{I_r & 0 \\ 0 & 0}$(幂等矩阵结构定理).幂等矩阵具有如下性质:
\begin{enumerate}
    \begin{multicols}{2}
        \item $A(I-A)=(I-A)A=0$
        \item $N(A)=C(I-A)$
        \item $N(I-A)=C(A)$
        \item $C^n=C(A)\oplus N(A)$
    \end{multicols}
\end{enumerate}
\begin{remarkbox}
    零(Null)空间: $N(A)=\{x|Ax=0,x\in C^n\}$\\
    列空间: $C(A)=\{Ay|y\in C^n\}$ (列向量张成的空间) \\
    行空间: $C(A^\HH),\qquad$
    左零空间: $N(A^\HH)$\\
    一个矩阵的零空间和行空间是正交的,左零空间与列空间是正交的.\\
    \sout{有的时候列空间(range space, 值域)也可以表示为$R(A)$,有些坑人.}
\end{remarkbox}

\textbf{投影变换}: 将$n$维酉空间$V$分解为两个子空间$S,T$,且$V=S\oplus T$, 则$V$中任一向量都可以唯一表示为
$$\al=x+y,\quad x\in S, y\in T$$
则称线性变换$$\tau(\al)=x$$为$V$沿$T$到$S$的投影变换.
\begin{remarkbox}
    $\tau$是投影变换等价于$\tau$的矩阵是幂等矩阵,\\
    幂等矩阵$A$对应了$C^n$沿着$N(A)$到$C(A)$的投影变换\\
    这从几何上来说, 代表多次投影后的结果保持不变.
\end{remarkbox}

\textbf{正交投影变换}: 如果两个子空间$S,T$中任意两个向量内积均为零,则称两个子空间是正交的,此时他们的直和加强为正交和,投影变换加强为\emph{正交投影变换}.
\begin{remarkbox}
    正交投影变换等价于矩阵是幂等H-矩阵$A$, 增加H条件后列空间与行空间相等了.\\
    进一步等价于$A=U_1U_1^\HH$\\
    其中$U_1$为次酉矩阵(从酉矩阵中保留若干列得到).
\end{remarkbox}

\subsection{正规矩阵}
\textbf{Schur引理}: 任何一个$n$阶复矩阵$A$都酉相似于一个上(下)三角矩阵. (数学归纳法证明)

\textbf{Schur不等式}: 设矩阵$A=(a_{ij})\in C^{n\times n}$, $\la_i$为特征值,则\begin{equation}
    \sum_{i=1}^{n}|\la_i|^2\le\sum_{i,j}|a_{ij}|^2
\end{equation}当且仅当$A$酉相似于对角矩阵是等号成立.

\textbf{正规矩阵}: 满足$AA^\HH=A^\HH A$的矩阵$A$称为正规矩阵. 其\emph{充要条件}是有相似于对角矩阵,其中对角线上值为特征值(正规矩阵结构定理).
\begin{remarkbox}
    结构定理的证明:对于正规矩阵$A$,和$A$酉相似的矩阵一定是正规矩阵;\\如果正规矩阵$A$是三角矩阵,则其必为对角矩阵;\\正规矩阵酉相似于一个正规三角矩阵.
    \\[0.7em]
    $n$阶正规矩阵一定有$n$个线性无关的特征向量,同时不同特征值的特征向量彼此正交.
\end{remarkbox}

正规矩阵同时酉对角化: 当且仅当$AB=BA$时, 两个矩阵可以同时被酉对角化.
\begin{remarkbox}
    首先将$A$使用$U_1$酉对角化为$C$,$B$也同时做相应变换为$D$;\\
    由于$AB=BA\Rightarrow CD=DC$的条件,$D$实际是由正规矩阵块组成的准对角线矩阵;\\
    相应的就有一个由酉矩阵块组成的准对角线矩阵$V$,$C$经过$V$的变换能够保持不变;\\
    最终两个矩阵被$U_1V$同时酉对角化.
\end{remarkbox}

\subsection{Hermite二次型}
\textbf{H-阵结构定理}: H-阵酉相似于实对角矩阵

\textbf{\important 正定矩阵}: 对于不全为零的向量$X$,$X^\HH AX>0$,则矩阵$A$正定.正定矩阵具有如下等价命题:
\begin{multicols}{2}
    \begin{enumerate}
        \item 对于任何$n$阶可逆矩阵,$P^\HH AP$均正定
        \item $A$的特征值均大于0
        \item 存在可逆矩阵$P$使得$P^\HH AP=I$
        \item 存在可逆矩阵$Q$使得$A=Q^\HH Q$
        \item 存在正线上三角矩阵$R$使得$A$被唯一分解为$A=R^\HH R$
    \end{enumerate}
\end{multicols}

类似地,半正定矩阵有如下等价命题:
\begin{enumerate}
    \item 对于任何$n$阶可逆矩阵,$P^\HH AP$半正定
    \item $A$的特征值非负
    \item 存在可逆矩阵$P$使得$P^\HH AP=\mat{I_r&0\\0&0}$
    \item 存在秩为$r$的矩阵$Q$使得$A=Q^\HH Q$
\end{enumerate}
\begin{remarkbox}
    正定矩阵可以依靠顺序主子式均大于0判断\\
    负定矩阵的顺序主子式符号奇数次小于0,偶数次大于0\\
    半正定(半负定)无法依靠顺序主子式判断,他们需要判断所有主子式.
    \begin{remarkbox}
        主子式: 提取矩阵相同索引的一组行和列组成的子式\\
        顺序主子式: 前$ k $行和前$ k $列形成的主子式
    \end{remarkbox}
\end{remarkbox}

H-矩阵偶在复合同下的标准形: 若$A,B$为H-阵,其中$B$正定,则存在可逆矩阵使得\begin{equation*}
    P^\HH AP=\mat{\la_1\\&\la_2\\&&\ddots\\&&&\la_n},\qquad P^\HH BP=I
\end{equation*}其中$\la_1,\la_2,\cdots,\la_n$是$|\la B-A|$的根,被称为\emph{$A$相对于$B$的广义特征值},与$P$无关.相应地满足\begin{equation*}
    AX=\la BX
\end{equation*}的$\la$被称为对应的\emph{广义特征向量}.


\subsection{RayLeigh商}
H-矩阵$A$的Rayleigh商定义为:\begin{equation}
    R(X)=\frac{X^\HH AX}{X^\HH X}(X\in C^n, X\ne0)
\end{equation}具有性质如下:
\begin{multicols}{2}
    \begin{enumerate}
        \item $R(kX)=R(X)$
        \item $\la_1\le R(X)\le\la_n$
        \item $\displaystyle\min_{X\ne0}R(X)=\la_1$
        \item $\displaystyle\max_{X\ne0}R(X)=\la_n$
    \end{enumerate}
\end{multicols}


\section{矩阵的分解}

\subsection{满秩分解}
对矩阵$A$做初等行列变换化成如下形式\begin{equation*}
    PAQ=\begin{bmatrix}
        I_r & D \\ 0 & 0
    \end{bmatrix}
\end{equation*}则矩阵$A$可以表示为一个列满秩矩阵与行满秩矩阵的乘积\begin{equation}
    A=\underbrace{P^{-1}\mat{I_r \\ 0}}_B \underbrace{\mat{I_r & D}Q^{-1}}_C = BC
\end{equation}满秩分解不是唯一的,可以选择不同的线性无关组来组成.
\begin{remarkbox}
    满秩分解的一个简单操作步骤:
    \begin{enumerate}
        \item 先只靠行变换变为最简阶梯型
        \item 在原矩阵中提取所有主元所在列序号的列向量,组成$B$
        \item 最简阶梯型去除末端的全0行,即得到$C$
    \end{enumerate}
\end{remarkbox}

\subsection{正交三角分解}
对于$n$阶可逆矩阵,其可被唯一分解为$$A=UR,\qquad A=R_1U_1$$其中$R$是正线上三角矩阵,$R_1$是正线下三角矩阵,$U,U_1$是酉矩阵.
分解方法为:
\begin{enumerate}
    \item 将矩阵$A$按列分块为$n$个线性无关的列向量${\al_1,\al_2,\cdots,\al_n}$
    \item 标准正交化后得到${\eta_1,\eta_2,\cdots,\eta_n}$
    \item $A=\mat{\al_1& \al_2& \cdots& \al_n}=\mat{\eta_1& \eta_2& \cdots& \eta_n}
              \begin{bmatrix}
                  c_{11} & c_{21} & \cdots & c_{n1} \\
                         & c_{22} & \cdots & c_{n2} \\
                         &        & \ddots & \vdots \\
                         &        &        & c_{nn}
              \end{bmatrix}=UR$
\end{enumerate}
\begin{remarkbox}
    其中矩阵为三角形是由正交化过程\eqref{eq:schmidt}保证的.
\end{remarkbox}

若矩阵$A$列满秩,则其可被唯一分解为\begin{equation*}
    A=UR
\end{equation*}若其行满秩,则可被唯一分解为\begin{equation*}
    A=LU
\end{equation*}上面两式中$U$为次酉矩阵,$R$为正线上三角矩阵,$L$为正线下三角矩阵.

\subsection{奇异值分解}
对于任意矩阵$A$, $AA^\HH$与$A^\HH A$都是半正定H-矩阵,同时有\begin{equation*}
    \Rank{A^\HH A}=\Rank{AA^\HH}=\Rank{A}
\end{equation*}同时两者的非零特征值相等,称他们\emph{正特征值}的平方根\begin{equation*}
    \al_i=\sqrt{\la_i}=\sqrt{\mu_i}>0
\end{equation*}为矩阵$A$的正奇异值,简称为奇异值.
\begin{remarkbox}
    正规矩阵的奇异值是非零特征值的模长\\
    酉矩阵特征值模长为1,奇异值为1
\end{remarkbox}

\textbf{奇异值分解计算方法}:
\begin{enumerate}
    \item 求出$AA^\HH$的$m$个特征值以及$r$个非零特征值,记为$\la_1\ge\la_2\ge\cdots\ge\la_r>\la_{r+1}=\la_{r+2}=\cdots=\la_m=0$
    \item 计算奇异值$\delta_1,\delta_2,\cdots,\delta_r$,得到$\Delta=\mr{diag}(\delta_1,\delta_2,\cdots,\delta_r)$
    \item 求出$AA^\HH$对应各特征值的标准正交特征向量$\eta_1,\eta_2,\cdots,\eta_m$,得到矩阵$$U=\mat{\eta_1,\eta_2,\cdots,\eta_r,\cdots,\eta_m}=\mat{U_1&U_2}$$
    \item \important $V_1=A^\HH U_1\Delta^{-\HH}$, $V_1$是次酉矩阵,构造$V_2$使得$V=\mat{V_1&V_2}$为酉矩阵
    \item 得到分解式$$A=U\mat{\Delta&0\\0&0}V^\HH$$
\end{enumerate}
\begin{remarkbox}
    \begin{enumerate}
        \item 奇异值分解不唯一
        \item $U$的列向量是$AA^\HH$的标准正交特征向量;$V$的列向量是$A^\HH A$的标准正交列向量
        \item 视计算量,可以先分解转置矩阵后再转置回来
    \end{enumerate}
\end{remarkbox}


\subsection{极分解}
对于$n$阶可逆矩阵$A$,其可被唯一分解为$$A=H_1U=UH_2$$其中$U$是酉矩阵,$H_1,H_2$是正定H-矩阵,同时有如下关系:$$A^\HH A=H_2^2,\quad AA^\HH=H_1^2$$
分解方法为:\begin{align*}
    H_2^2 & =A^\HH A,\qquad U=AH_2^{-1} \\
    H_1   & =UH_2U^\HH
\end{align*}
\begin{remarkbox}
    如果$A$不可逆,仍然可以进行分解,只不过$H$矩阵的条件从正定放宽到半正定.\\[0.7em]
    分解方法的第一个等式利用了(半)正定矩阵的平方根, $U$是酉矩阵可以由第一个等式变形简单验证.
\end{remarkbox}

\subsection{谱分解}
\textbf{正规矩阵的谱分解}: 根据正规矩阵的结构定理,存在酉矩阵$U$,使得正规矩阵$A$被分解为:\begin{equation*}
    A=U\Lambda U^\HH=\la_1\al_1\al_1^\HH + \la_2\al_2\al_2^\HH + \cdots + \la_n\al_n\al_n^\HH
\end{equation*}其中多重特征值还可以进一步合并,表示为\begin{equation*}
    A=\sum_{i=1}^{r}\la_i\sum_{j=1}^{n_i}\al_{i_j}\al_{i_j}^\HH=\sum_{i=1}^{r}\la_iG_i
\end{equation*}

\textbf{单纯(可对角化)矩阵的谱分解}: 将$A$相似对角化的结果变形得:\begin{align*}
    A=P\Lambda P^{-1} & =\mat{\al_1,\al_2,\cdots,\al_n}
    \begin{bmatrix}
        \la_1                     \\
         & \la_2                  \\
         &       & \ddots         \\
         &       &        & \la_n
    \end{bmatrix}
    \begin{bmatrix}
        \be_1^\TT \\\be_2^\TT\\\vdots\\\be_n^\TT
    \end{bmatrix}             \\& = \sum_{i=1}^{n}\la_i\al_i\be_i^\TT = \sum_{i=1}^{r}\la_iG_i
\end{align*}


\section{向量与矩阵范数}

\subsection{向量范数}
向量范数需要满足三个性质:
\begin{enumerate}
    \item 非负性: $\norm{\al}\ge 0$, 当且仅当$\al=0$时$\norm{\al}=0$
    \item 齐次性: $\norm{k\al}=|k|\norm{\al}$
    \item 三角不等式: $\norm{\al+\be}\le\norm{\al}+\norm{\be}$
\end{enumerate}
常用的三个向量范数: $\norm{\al}_1=\displaystyle\sum_{i=1}^{n}|a_i|;$
$\norm{\al}_2=\displaystyle\sqrt{\sum_{i=1}^{n}|a_i|^2}=\sqrt{\al^\HH\al};$
$\norm{\al}_\infty=\displaystyle\max_{1\le i\le n}|a_i|$

\begin{remarkbox}
    常用不等式1(Holder不等式):\begin{equation*}
        \sum_{i=1}^{n}|a_ib_i|\le\left(\sum_{i=1}^{n}\left|a_i\right|^p\right)^{1/p}\left(\sum_{i=1}^{n}\left|b_i\right|^p\right)^{1/p}
    \end{equation*}其中$p>1,q>1$且$1/p+1/q=1$\\[1em]
    常用不等式2(Minkowski不等式):\begin{equation*}
        \left(\sum_{i=1}^{n}\left|a_i+b_i\right|^p\right)^{1/p}\le\left(\sum_{i=1}^{n}\left|a_i\right|^p\right)^{1/p}+\left(\sum_{i=1}^{n}\left|b_i\right|^p\right)^{1/p}
    \end{equation*}其中$p\ge1$
\end{remarkbox}

向量范数的等价性: 对于两个有限维线性空间$V$上的向量范数$\norm{\al}_a,\norm{\al}_b$, 存在两个与$\al$无关的正数使得\begin{equation*}
    d_1\norm{\al}_b\le \norm{\al}_a\le d_2\norm{\al}_b, \qquad \forall \al\in V
\end{equation*}

利用向量范数构造新范数: 设$\norm{\cdot}_b$是$C^m$上的向量范数,$A\in C^{m\times n}_n$, 则\begin{equation*}
    \norm{\al}_a = \norm{A\al}_b
\end{equation*}所定义的$\norm{\cdot}_a$是$C^n$上的向量范数.

\subsection{矩阵范数}
矩阵范数需要满足四个性质:
\begin{enumerate}
    \item 非负性: $\norm{A}\ge 0$, 当且仅当$A=0$时$\norm{A}=0$
    \item 齐次性: $\norm{kA}=|k|\norm{A}$
    \item 三角不等式: $\norm{A+B}\le\norm{A}+\norm{B}$
    \item \important 矩阵乘法相容性: $\norm{AB}\le\norm{A}\norm{B}$
\end{enumerate}
矩阵范数也有类似于向量范数的等价性,不再赘述.

常见矩阵范数:
\begin{enumerate}
    \item Frobenious范数 $\norm{A}_F^2=\displaystyle\sum_{i=1}^{m}\sum_{i=1}^{n}\left|a_{ij}\right|^2=\Tr{A^\HH A}$
    \item 列和范数 $\norm{A}_1=\displaystyle\max_j\left(\sum_{i=1}^{m}\left|a_{ij}\right|\right)$
    \item 谱范数 $\norm{A}_2=\displaystyle\max_{j}\sqrt{\la_j\left(A^\HH A\right)}$
    \item 行和范数 $\norm{A}_\infty=\displaystyle\max_i\left(\sum_{j=1}^{n}\left|a_{ij}\right|\right)$
\end{enumerate}
\begin{remarkbox}
    Frobenious范数的一些其他性质:
    \begin{enumerate}
        \item $\displaystyle A=\mat{\al_1&\al_2&\cdots&\al_n}, \norm{A}_F^2=\sum_{i=1}^{n}\norm{\al_i}_2^2$
        \item 酉不变性: $\norm{A}_F=\norm{UA}_F=\norm{A^\HH}_F=\norm{AV}_F=\norm{UAV}_F$
        \item Frobenious范数与向量2-范数\emph{相容}
    \end{enumerate}
\end{remarkbox}

\textbf{矩阵范数与向量范数相容}: 对于任意矩阵$A$与向量$X$都有\begin{equation*}
    \norm{AX}_\al\le\norm{A}_\be\norm{X}_\al
\end{equation*}则称矩阵范数$\norm{A}_\be$与向量范数$\norm{X}_\al$是相容的.

\textbf{诱导范数(算子范数)}: 设$\norm{X}_\al$是向量范数,则\begin{equation*}
    \norm{A}_i=\max_{X\ne0}\frac{\norm{AX}_\al}{\norm{X}_\al}
\end{equation*}是与$\norm{X}_\al$ \emph{相容}的矩阵范数,被称为算子范数.

\textbf{由矩阵范数构造向量范数}: 设$\norm{A}_*$是矩阵范数,则存在与其\emph{相容}的向量范数$\norm{X}=\norm{X\al^\HH}_*$.

\textbf{\important 谱半径}: 矩阵$A$所有特征值的绝对值中的最大值被称为矩阵$A$的谱半径,即\begin{equation*}
    \rho(A)=\max\left\{|\la_1|,|\la_2|,\cdots,|\la_n|\right\}
\end{equation*}矩阵$A$的任意一种范数都不小于它的谱半径,即$\rho(A)\le\norm{A}$.

\subsection{矩阵序列与极限}
矩阵各项收敛即可得到矩阵序列收敛, 除加法数乘外还有如下运算规则:
\begin{enumerate}
    \item $\displaystyle\lim_{k\to\infty}A^{(k)}B^{(k)}=AB$
    \item $\displaystyle\lim_{k\to\infty}PA^{(k)}Q=PAQ$
    \item $\displaystyle\lim_{k\to\infty}\left(A^{(k)}\right)^{-1}=A^{-1}$
\end{enumerate}

矩阵序列$\left\{A^{(k)}\right\}$收敛于$A$的充要条件为$\displaystyle\lim_{k\to\infty}\norm{A^{(k)}-A}=0$.
\begin{remarkbox}
    根据矩阵范数等价性,任意范数成立上式后都可以使用夹逼准则证明另一范数满足该条件,因此证明时可以使用最简单形式的范数$\norm{A}=\displaystyle\sum_{i=1}^{m}\sum_{j=1}^{n}|a_{ij}|$.
\end{remarkbox}

矩阵序列$\left\{A,A^2,\cdots,A^k,\cdots\right\}$收敛于$0$的充要条件为$\rho(A)<0$.

\subsection{矩阵幂级数}
矩阵级数$\displaystyle\sum_{k=1}^{\infty}A^{(k)}$绝对收敛的充要条件是正项级数$\displaystyle\sum_{k=1}^{\infty}\norm{A^{(k)}}$收敛.

矩阵幂级数$\displaystyle\sum_{k=0}^{\infty}a_kA^k$在$\rho(A)<R$时绝对收敛,在$\rho(A)>R$时发散,在$\rho(A)=R$时需额外判断.其中$R$为收敛半径,计算公式如下\begin{equation*}
    \frac{1}{R}=\lim_{k\to\infty}\left|\frac{a_{k+1}}{a_k}\right|
\end{equation*}
\begin{remarkbox}
    因为谱半径是所有范数值的下界,因此可以通过寻找在收敛域内的范数值来说明级数收敛.
\end{remarkbox}

\textbf{\important Jordan块矩阵的幂}: \begin{equation}
    J_i^k(\la_i)=\begin{bmatrix}
        \la_i^k & C^1_k\la_i^{k-1} & \cdots & C_k^{d_i-1}\la_i^{k-d_i+1} \\
                & \la_i^k          & \ddots & \vdots                     \\
                &                  & \ddots & C^1_k\la_i^{k-1}           \\
                &                  &        & \la_i^k
    \end{bmatrix}_{d_i\times d_i}
\end{equation}
\begin{remarkbox}
    常见函数的幂级数展开, 前三个全域收敛, 后三个收敛半径为$1$
    \begin{multicols}{2}
        \begin{enumerate}
            \item $e^x=\displaystyle\sum_{n=0}^{\infty}\frac{x^n}{n!}$
            \item $\sin x=\displaystyle\sum_{n=0}^{\infty}\frac{(-1)^n}{(2n+1)!}x^{2n+1}$
            \item $\cos x=\displaystyle\sum_{n=0}^{\infty}\frac{(-1)^n}{(2n)!}x^{2n}$
            \item $\arctan x=\displaystyle\sum_{n=1}^{\infty}\frac{(-1)^{n-1}}{2n-1}x^{2n-1}$
            \item $\ln x=\displaystyle\sum_{n=1}^{\infty}\frac{(-1)^{n+1}}{n}x^{n}$
            \item $\displaystyle\frac{1}{1+x}=(1+x)^{-1}=\sum_{i=0}^{\infty}(-1)^nx^n$
        \end{enumerate}
    \end{multicols}
\end{remarkbox}

一个特殊级数:矩阵幂级数\begin{equation*}
    I+A+A^2+\cdots+A^k+\cdots
\end{equation*}绝对收敛的充要条件是$\rho(A)<1$,收敛和为$(I-A)^{-1}$.
\begin{remarkbox}
    充分性证明是前文定理,必要性证明是$\lim_{k\to\infty}\norm{A^k}=0$.
\end{remarkbox}


\section{矩阵函数}
\newcommand{\ddd}[2]{\dfrac{\dd #1}{\dd #2}}
\newcommand{\ddx}[1]{\ddd{#1}{x}}
\newcommand{\ddt}[1]{\ddd{#1}{t}}

\subsection{矩阵多项式}
$\displaystyle p(A)=\sum_{i=0}^{m}a_iA^i$被称为矩阵多项式. 考虑$d_i$阶Jordan块矩阵$J_i$,则\begin{equation}
    p(J_i)=\begin{bmatrix}
        p(\la_i) & p'(\la_i) & \frac{p''(\la_i)}{2!} & \cdots & \frac{p^{(d_i-1)}(\la_i)}{(d_i-1)!} \\
                 & p(\la_i)  & p'(\la_i)             & \ddots & \vdots                              \\
                 &           & p(\la_i)              & \ddots & \frac{p''(\la_i)}{2!}               \\
                 &           &                       & \ddots & p'(\la_i)                           \\
                 &           &                       &        & p(\la_i)
    \end{bmatrix}_{d_i\times d_i}
\end{equation}
\begin{remarkbox}
    注意其中的阶乘
\end{remarkbox}

\textbf{\important 矩阵多项式的Jordan表示:} $p(A)=P\mr{diag}(p(J_1),p(J_2),\cdots,p(J_r))P^{-1}$, 其中$P$是将$A$变换为Jordan标准形的变换矩阵.
\begin{remarkbox}
    矩阵多项式$p(A)$的特征值对应为$p(\la_i)$,对应的特征向量不变. (使用定义证明)
\end{remarkbox}

\textbf{化零多项式:} 满足$f(A)=O_{n\times n}$的多项式,被称为矩阵$A$的化零多项式. 根据Hamilton-Cayley定理, \emph{矩阵$A$的特征多项式$\mr{det}(\la E-A)$是化零多项式}.

\textbf{最小多项式:} 次数最低且首项系数为1的化零多项式为最小多项式,记为$m(\la)$. 性质如下:
\begin{enumerate}
    \item 任何一个化零多项式都能被$m(\la)$整除
    \item 最小多项式唯一
    \item 相似矩阵具有相同的最小多项式
\end{enumerate}
最小多项式的求法如下:
\begin{enumerate}
    \item 化为Jordan标准形,相似矩阵最小多项式相同,只需考虑标准形
    \item 对于$d_i$阶Jordan块矩阵,简单运算一下即可得到其最小多项式为$(\la-\la_i)^{d_i}$
    \item 准对角矩阵的最小多项式为各块最小多项式的最大公倍式
\end{enumerate}

\subsection{矩阵函数}
\textbf{定义:} $A$的最小多项式为\begin{equation*}
    m(\la)=\sum_{i=1}^{s}(\la-la_i)^{d_ii}  \qquad  \left(\sum_{i=1}^{s}d_i=m\right)
\end{equation*}如果函数$f(x)$具有足够高阶的导数并且以下$m$个值
$$\{f(\la_i),f'(\la_i),\cdots,f^{(d_i-1)}(\la_i),i=1,2,\cdots,s\}$$存在,则称\emph{$f(x)$在矩阵$A$的影谱上有定义}.

$f(x)$在影谱上有定义,如果存在多项式$p(\la)$的上面$m$个值与$f(\la)$相等,则定义矩阵函数$f(A)=p(A)$. 满足定义的多项式\emph{存在但并不唯一}, 只要影谱上对应的值相等, 最终矩阵函数就相等.

\textbf{\important 矩阵函数的Jordan表示:} 与上文矩阵多项式的Jordan表示是相似的,不过是把多项式换为了函数,同时要求在影谱上有定义.

\textbf{\important 矩阵函数的多项式表示:} 使用拉格朗日插入计算多项式系数$a_0,a_1,\cdots,a_{m-1}$,使得多项式影谱上对应值相等\begin{equation*}
    p^{(k)}(\la_i)=f^{(k)}(\la_i), \quad i=1,2,\cdots,s; \quad k=1,2,\cdots,d_i-1
\end{equation*}最终可得到多项式表示$f(A)=a_{m-1}A^{m-1}+\cdots+a_1A+A_0I$

\textbf{\important 矩阵函数的幂级数表示:} 将一元函数$f(x)$展开为收敛半径$R$的幂级数$\displaystyle\sum_{k=0}^{\infty}c_kx^k$. 当矩阵$A$的谱半径$\rho(A)<R$时,矩阵幂级数$\displaystyle\sum_{k=0}^{\infty}c_kx^k$绝对收敛, 该级数即为急诊的幂级数表示.

\subsection{矩阵指数函数和矩阵三角函数}

\begin{align}
    e^{At}  & = \sum_{k=0}^{\infty}\frac{1}{k!}A^Kt^k                     \\
    \sin At & = \sum_{k=0}^{\infty}\frac{(-1)^k}{(2k+1)!}A^{2k+1}t^{2k+1} \\
    \cos At & = \sum_{k=0}^{\infty}\frac{(-1)^k}{(2k)!}A^{2k}t^{2k}
\end{align}
\begin{remarkbox}
    复指数函数的一些结论: $e^{iA}=\cos A+i\sin A$, $e^{-iA}=\cos A-i\sin A$, \\[0.5em]
    $\cos A=\dfrac{e^{iA}+e^{-iA}}{2}$, $\sin A=\dfrac{e^{iA}-e^{-iA}}{2i}$
\end{remarkbox}

当$AB=BA$时,一些标量三角函数的结论可套用:
\begin{enumerate}
    \item $e^{A+B}=e^Ae^B=e^Be^A$ (若$AB$不可交换,该性质不一定成立,但也可能成立)
    \item 三角函数两角和, 二倍角, 平方和
\end{enumerate}

几个特殊性质如下:
\begin{enumerate}
    \item $\ddt{e^{At}}=Ae^{At}=e^{At}A$
    \item $\ddt{\sin At}=A(\cos At)=(\cos At)A$
    \item $\ddt{\cos At}=-A(\sin At)=-(\sin At)A$
    \item $|e^{At}|=e^{\Tr{A}}$
\end{enumerate}


\section{函数矩阵与矩阵微分方程}

\subsection{函数矩阵的导数与积分}
本部分并没有什么值得一提的东西,基本都是把多个标量函数摞在一起运算了一下. 需要注意一些由于矩阵乘法导致交换律不适用的运算:
\begin{enumerate}
    \item $\ddx{}\left[A(x)B(x)\right]=\ddx{A}B+A\ddx{B}$
    \item $\ddx{A^{-1}(x)}=-A^{-1}\ddx{A}A^{-1}$
\end{enumerate}
以及一些做题时经常用到的公式:
\begin{equation}
    \ddx{}\left(\int_{a(x)}^{b(x)}f(t)\dd t\right) = f(b(x))b'(x) - f(a(x))a'(x)
\end{equation}

\subsection{矩阵微分方程}
\sout{我看往年题都不考这里,先暂缓整理吧.}


\section{广义逆与最小二乘}

\subsection{广义逆}
广义逆矩阵$A^-$的充要条件为\begin{equation}\label{eq:inv1}
    AA^-A=A
\end{equation}具有如下性质:
\begin{enumerate}
    \item $A^-b$是方程组$Ax=b$的解 \textbf{(定义)}
    \item $(A^-)^\TT=(A^\TT)^-$
    \item $\Rank{A}\le\Rank{A^-}$
    \item $AA^-$和$A^-A$都是幂等矩阵
\end{enumerate}

\begin{remarkbox}
    根据定义就可以看出,广义逆是不一定唯一的
\end{remarkbox}

\textbf{定义:} 若$A_L^{-1}A=E_n$,则$A_L^{-1}$被称为左逆; 若$AA^{-1}_R=E_m$,则$A_R^{-1}$被称为右逆.
$A$列满秩的充要条件是$A^-A=E$, $A$行满秩的充要条件是$AA^-=E$.

\subsection{自反广义逆}
在\eqref{eq:inv1}的基础上附加另一条件\begin{equation}\label{eq:inv2}
    A^-AA^-=A^-
\end{equation}
满足该定义的广义逆成为自反广义逆,记作$A_r^-$,相比于广义逆具有如下额外性质:
\begin{enumerate}
    \item $\Rank{A}=\Rank{A^-}$ \textbf{(充要条件)}
    \item 若$X,Y$是$A$的广义逆,则$XAY$是自反广义逆
    \item 若$A=BC$为满秩分解,则$C^{-1}_RB^{-1}_L$是自反广义逆
\end{enumerate}

\subsection{伪逆}
在\eqref{eq:inv1}\eqref{eq:inv2}的基础上额外添加两个对称条件,构成Penros-Moore方程如下:\begin{equation}
    \left\{
    \begin{aligned}
        AA^+A                 & =A    \\ A^+AA^+&=A^+                \\
        \left(AA^+\right)^\HH & =AA^+ \\ \left(A^+A\right)^\HH&=A^+A
    \end{aligned}
    \right.
\end{equation}
满足上述四个条件的特殊自反广义逆矩阵即为\textbf{伪逆矩阵},具有如下性质:
\begin{enumerate}
    \item 若$A=BC$是满秩分解,则$C^\HH\left(CC^\HH\right)^{-1}\left(B^\HH B\right)^{-1}B^\HH$是伪逆矩阵 \important
    \item $A^+=U\Lambda^+U^\HH A^\HH$, 其中$U^\HH A^\HH AU=\mr{diag}\{\la_1,\la_2,\cdots,\la_n\}=\Lambda$
    \item $\left(A^+\right)^+=A$
    \item $\left(AA^\HH\right)^+=\left(A^\HH\right)^+A^+=\left(A^+\right)^\HH A^+$, $\left(A^\HH A\right)^+=A^+\left(A^\HH\right)^+=A^+\left(A^+\right)^\HH $
    \item $A^+=A^\HH\left(AA^\HH\right)^+=\left(A^\HH A\right)^+A^\HH$
\end{enumerate}
\begin{remarkbox}
    伪逆是唯一的. (思路:$X=XAX=...=XAY=...=YAY=Y$)
\end{remarkbox}

\subsection{线性方程组}
矩阵方程$AXB=D$有解的充要条件为存在广义逆$A^-,B^-$使得\begin{equation}
    AA^-DB^-B=D
\end{equation}成立. 矩阵方程的通解为\begin{equation}
    X=A^-DB^-+Y-A^-AYBB^-
\end{equation}其中$Y$为任意符合尺寸的矩阵.

\textbf{推论1:} 矩阵方程$AX=D$有解的充要条件为存在广义逆$A^-$使得\begin{equation}
    AA^-D=D
\end{equation}成立. 矩阵方程的通解为\begin{equation}
    X=A^-D+Y-A^-AY
\end{equation}其中$Y$为任意符合尺寸的矩阵.

\textbf{推论2\important :} 矩阵方程$AX=b$有解的充要条件为存在广义逆$A^-$使得\begin{equation}
    AA^-b=b
\end{equation}成立. 矩阵方程的通解为\begin{equation}
    X=A^-b+(E_n-A^-A)Y
\end{equation}其中$Y$为任意符合尺寸的向量.

\textbf{定义:} 相容(有解)方程组$Ax=b$的所有解中模(2-范数)最小的解称为最小模解.

\textbf{定义:} 若$n$维向量$x_0$满足对于任何一个$n$维向量$x$,都有$\norm{Ax_0-b}^2\le\norm{Ax-b}^2$,则称$x_0$是方程组$Ax=b$的一个最小二乘解.

\textbf{定义:} 若对于任意一个最小二乘解$x_0$,最小二乘解$u$都满足$\norm{u}\le\norm{x_0}$,则称$u$为最佳最小二乘解.

最小二乘解有如下性质:
\begin{enumerate}
    \item 对于满足P-M方程\{1,3\}的广义逆矩阵$B$,$x=Bb$是$Ax=b$的最小二乘解
    \item $x=A^+b$是$Ax=b$的最佳最小二乘解 \important
    \item $\norm{Ax^*-b}_2$是$b$到$R(A)$的最短距离,其中$x^*$为最佳最小二乘解 \important
\end{enumerate}


\section{Kronecker积}
\subsection{基本性质}
\newcommand{\OT}[2]{#1 \otimes #2}
\newcommand{\OTL}[2]{\left(#1\right) \otimes #2}
\newcommand{\OTR}[2]{#1 \otimes \left(#2\right)}
\newcommand{\OTLR}[2]{\left(#1\right) \otimes \left(#2\right)}

$\OT{A}{B}=\mat{a_{11}B & a_{12}B & \cdots & a_{1n}B \\
        a_{21}B & a_{22}B & \cdots & a_{2n}B \\
        \vdots & \vdots & \ddots & \vdots \\
        a_{m1}B & a_{m2}B & \cdots & a_{mn}B \\}$, 有如下性质:
\begin{enumerate}
    \item 无交换律. 有结合律、分配律,数乘可任意移动位置
    \item $\left(\OT{A}{B}\right)\left(\OT{C}{D}\right) = \OT{AC}{BD}$  \important
    \item $\left(\OT{A}{B}\right)^\HH = \OT{A^\HH}{B^\HH}$
    \item $\left(\OT{A}{B}\right)^{-1} = \OT{A^{-1}}{B^{-1}}$
    \item $\Tr{\OT{A}{B}} = \Tr{A}\cdot\Tr{B}$
    \item $\Rank{\OT{A}{B}} = \Rank{A}\cdot\Rank{B}$
    \item $|\OT{A}{B}| = {|A|}^p{|B|}^m$ ($p,m$为阶数)
    \item 若$A,B$为正规矩阵、对称矩阵、H矩阵、反H矩阵、酉矩阵, $\OT{A}{B}$保持对应性质
    \item 存在合同变换,使得$P^\TT\left(\OT{A}{B}\right)P=\OT{B}{A}$, 其中$P$为有限个初等矩阵乘积
    \item $\OT{A}{B}\sim\OT{B}{A}$ (上面一条的$P$是正交矩阵)
    \item Kronecker积同样有幂的概念,$\left(AB\right)^{[k]}=A^{[k]}B^{[k]}$
\end{enumerate}

\subsection{特征值}
考虑$\displaystyle f(x,y)=\sum_{i,j}c_{ij}x^iy^j$,
则$\displaystyle f(A,B)=\sum_{i,j}c_{ij}A^i\otimes B^j$的特征值为$f(\la_r,\mu_s)$, 对应特征向量为 $\OT{x_r}{y_s}$,
(其中$\la_i,x_i$为$A$的特征值与特征向量, $\mu_j,y_j$为$B$的...)

\textbf{推论}: $\OT{A}{E_n}+\OT{E_m}{B}$被称为Kronecker和,特征值为$\la_r+\mu_s$

\subsection{行展开与列展开}
\newcommand{\rs}[1]{\mr{rs}\left(#1\right)}
\newcommand{\cs}[1]{\mr{cs}\left(#1\right)}
将矩阵$A$的每一行串接为一个行向量,成为$A$的行展开,记为$\mr{rs}(A)$.
同理,将矩阵$A$的每一列串接为一个列向量,成为$A$的列展开,记为$\mr{cs}(A)$,有如下性质:
\begin{enumerate}
    \item $\rs{A^\TT}=\left(\cs{A}\right)^\TT$,  $\cs{A^\TT}=\left(\rs{A}\right)^\TT$
    \item $\rs{ABC}=\rs{B}\left(\OT{A^\TT}{C}\right)$
    \item $\cs{ABC}=\left(\OT{C^\TT}{A}\right)\cs{B}$
    \item $\cs{AX}=\left(\OT{E_n}{A}\right)\cs{X}$
    \item $\cs{XB}=\left(\OT{B^\TT}{E_m}\right)\cs{X}$
\end{enumerate}


\end{document}