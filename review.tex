\documentclass{matrix-review}
\usepackage[normalem]{ulem}

\begin{document}

\section{1}
\subsection{正规矩阵}




\section{函数矩阵与矩阵微分方程}
\newcommand{\ddd}[2]{\dfrac{\dd #1}{\dd #2}}
\newcommand{\ddx}[1]{\ddd{#1}{x}}

\subsection{函数矩阵的导数与积分}
本部分并没有什么值得一提的东西,基本都是把多个标量函数摞在一起运算了一下. 需要注意一些由于矩阵乘法导致交换律不适用的运算:
\begin{enumerate}
    \item $\ddx{}\left[A(x)B(x)\right]=\ddx{A}B+A\ddx{B}$
    \item $\ddx{A^{-1}(x)}=-A^{-1}\ddx{A}A^{-1}$
\end{enumerate}
以及一些做题时经常用到的公式:
\begin{equation}
    \ddx{}\left(\int_{a(x)}^{b(x)}f(t)\dd t\right) = f(b(x))b'(x) - f(a(x))a'(x)
\end{equation}

\subsection{矩阵微分方程}
\sout{我看往年题都不考这里,先暂缓整理吧.}

\section{广义逆与最小二乘}

\subsection{广义逆}
广义逆矩阵$A^-$的充要条件为\begin{equation}\label{eq:inv1}
    AA^-A=A
\end{equation}具有如下性质:
\begin{enumerate}
    \item $A^-b$是方程组$Ax=b$的解 \textbf{(定义)}
    \item $(A^-)^\TT=(A^\TT)^-$
    \item $\Rank{A}\le\Rank{A^-}$
    \item $AA^-$和$A^-A$都是幂等矩阵
\end{enumerate}

\begin{remarkbox}
    根据定义就可以看出,广义逆是不一定唯一的
\end{remarkbox}

\textbf{定义:} 若$A_L^{-1}A=E_n$,则$A_L^{-1}$被称为左逆; 若$AA^{-1}_R=E_m$,则$A_R^{-1}$被称为右逆.
$A$列满秩的充要条件是$A^-A=E$, $A$行满秩的充要条件是$AA^-=E$.

\subsection{自反广义逆}
在\eqref{eq:inv1}的基础上附加另一条件\begin{equation}\label{eq:inv2}
    A^-AA^-=A^-
\end{equation}
满足该定义的广义逆成为自反广义逆,记作$A_r^-$,相比于广义逆具有如下额外性质:
\begin{enumerate}
    \item $\Rank{A}=\Rank{A^-}$ \textbf{(充要条件)}
    \item 若$X,Y$是$A$的广义逆,则$XAY$是自反广义逆
    \item 若$A=BC$为满秩分解,则$C^{-1}_RB^{-1}_L$是自反广义逆
\end{enumerate}

\subsection{伪逆}
在\eqref{eq:inv1}\eqref{eq:inv2}的基础上额外添加两个对称条件,构成Penros-Moore方程如下:\begin{equation}
    \left\{
    \begin{aligned}
        AA^+A                 & =A    \\ A^+AA^+&=A^+                \\
        \left(AA^+\right)^\HH & =AA^+ \\ \left(A^+A\right)^\HH&=A^+A
    \end{aligned}
    \right.
\end{equation}
满足上述四个条件的特殊自反广义逆矩阵即为\textbf{伪逆矩阵},具有如下性质:
\begin{enumerate}
    \item 若$A=BC$是满秩分解,则$C^\HH\left(CC^\HH\right)^{-1}\left(B^\HH B\right)^{-1}B^\HH$是伪逆矩阵 \important
    \item $A^+=U\Lambda^+U^\HH A^\HH$, 其中$U^\HH A^\HH AU=\mr{diag}\{\la_1,\la_2,\cdots,\la_n\}=\Lambda$
    \item $\left(A^+\right)^+=A$
    \item $\left(AA^\HH\right)^+=\left(A^\HH\right)^+A^+=\left(A^+\right)^\HH A^+$, $\left(A^\HH A\right)^+=A^+\left(A^\HH\right)^+=A^+\left(A^+\right)^\HH $
    \item $A^+=A^\HH\left(AA^\HH\right)^+=\left(A^\HH A\right)^+A^\HH$
\end{enumerate}
\begin{remarkbox}
    伪逆是唯一的. (思路:$X=XAX=...=XAY=...=YAY=Y$)
\end{remarkbox}

\subsection{线性方程组}
矩阵方程$AXB=D$有解的充要条件为存在广义逆$A^-,B^-$使得\begin{equation}
    AA^-DB^-B=D
\end{equation}成立. 矩阵方程的通解为\begin{equation}
    X=A^-DB^-+Y-A^-AYBB^-
\end{equation}其中$Y$为任意符合尺寸的矩阵.

\textbf{推论1:} 矩阵方程$AX=D$有解的充要条件为存在广义逆$A^-$使得\begin{equation}
    AA^-D=D
\end{equation}成立. 矩阵方程的通解为\begin{equation}
    X=A^-D+Y-A^-AY
\end{equation}其中$Y$为任意符合尺寸的矩阵.

\textbf{推论2\important :} 矩阵方程$AX=b$有解的充要条件为存在广义逆$A^-$使得\begin{equation}
    AA^-b=b
\end{equation}成立. 矩阵方程的通解为\begin{equation}
    X=A^-b+(E_n-A^-A)Y
\end{equation}其中$Y$为任意符合尺寸的向量.

\textbf{定义:} 相容(有解)方程组$Ax=b$的所有解中模(2-范数)最小的解称为最小模解.

\textbf{定义:} 若$n$维向量$x_0$满足对于任何一个$n$维向量$x$,都有$\norm{Ax_0-b}^2\le\norm{Ax-b}^2$,则称$x_0$是方程组$Ax=b$的一个最小二乘解.

\textbf{定义:} 若对于任意一个最小二乘解$x_0$,最小二乘解$u$都满足$\norm{u}\le\norm{x_0}$,则称$u$为最佳最小二乘解.

最小二乘解有如下性质:
\begin{enumerate}
    \item 对于满足P-M方程\{1,3\}的广义逆矩阵$B$,$x=Bb$是$Ax=b$的最小二乘解
    \item $x=A^+b$是$Ax=b$的最佳最小二乘解 \important
    \item $\norm{Ax^*-b}_2$是$b$到$R(A)$的最短距离,其中$x^*$为最佳最小二乘解 \important
\end{enumerate}

\section{Kronecker积}
\subsection{基本性质}
\newcommand{\OT}[2]{#1 \otimes #2}
\newcommand{\OTL}[2]{\left(#1\right) \otimes #2}
\newcommand{\OTR}[2]{#1 \otimes \left(#2\right)}
\newcommand{\OTLR}[2]{\left(#1\right) \otimes \left(#2\right)}

$\OT{A}{B}=\mat{a_{11}B & a_{12}B & \cdots & a_{1n}B \\
        a_{21}B & a_{22}B & \cdots & a_{2n}B \\
        \vdots & \vdots & \ddots & \vdots \\
        a_{m1}B & a_{m2}B & \cdots & a_{mn}B \\}$, 有如下性质:
\begin{enumerate}
    \item 无交换律. 有结合律、分配律,数乘可任意移动位置
    \item $\left(\OT{A}{B}\right)\left(\OT{C}{D}\right) = \OT{AC}{BD}$  \important
    \item $\left(\OT{A}{B}\right)^\HH = \OT{A^\HH}{B^\HH}$
    \item $\left(\OT{A}{B}\right)^{-1} = \OT{A^{-1}}{B^{-1}}$
    \item $\Tr{\OT{A}{B}} = \Tr{A}\cdot\Tr{B}$
    \item $\Rank{\OT{A}{B}} = \Rank{A}\cdot\Rank{B}$
    \item $|\OT{A}{B}| = {|A|}^p{|B|}^m$ ($p,m$为阶数)
    \item 若$A,B$为正规矩阵、对称矩阵、H矩阵、反H矩阵、酉矩阵, $\OT{A}{B}$保持对应性质
    \item 存在合同变换,使得$P^\TT\left(\OT{A}{B}\right)P=\OT{B}{A}$, 其中$P$为有限个初等矩阵乘积
    \item $\OT{A}{B}\sim\OT{B}{A}$ (上面一条的$P$是正交矩阵)
    \item Kronecker积同样有幂的概念,$\left(AB\right)^{[k]}=A^{[k]}B^{[k]}$
\end{enumerate}

\subsection{特征值}
考虑$\displaystyle f(x,y)=\sum_{i,j}c_{ij}x^iy^j$,
则$\displaystyle f(A,B)=\sum_{i,j}c_{ij}A^i\otimes B^j$的特征值为$f(\la_r,\mu_s)$, 对应特征向量为 $\OT{x_r}{y_s}$,
(其中$\la_i,x_i$为$A$的特征值与特征向量, $\mu_j,y_j$为$B$的...)

\textbf{推论}: $\OT{A}{E_n}+\OT{E_m}{B}$被称为Kronecker和,特征值为$\la_r+\mu_s$

\subsection{行展开与列展开}
\newcommand{\rs}[1]{\mr{rs}\left(#1\right)}
\newcommand{\cs}[1]{\mr{cs}\left(#1\right)}
将矩阵$A$的每一行串接为一个行向量,成为$A$的行展开,记为$\mr{rs}(A)$.
同理,将矩阵$A$的每一列串接为一个列向量,成为$A$的列展开,记为$\mr{cs}(A)$,有如下性质:
\begin{enumerate}
    \item $\rs{A^\TT}=\left(\cs{A}\right)^\TT$,  $\cs{A^\TT}=\left(\rs{A}\right)^\TT$
    \item $\rs{ABC}=\rs{B}\left(\OT{A^\TT}{C}\right)$
    \item $\cs{ABC}=\left(\OT{C^\TT}{A}\right)\cs{B}$
    \item $\cs{AX}=\left(\OT{E_n}{A}\right)\cs{X}$
    \item $\cs{XB}=\left(\OT{B^\TT}{E_m}\right)\cs{X}$
\end{enumerate}

\end{document}