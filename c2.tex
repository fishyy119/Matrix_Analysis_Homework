\documentclass{matrix-homework}
\newcommand{\la}{\lambda}
\newcommand{\simeqstep}[1]{\;\overset{#1}{\simeq}\;}
\newcommand{\rowSwap}[2]{\simeqstep{R_{#1} \leftrightarrow R_{#2}}}
\newcommand{\rowScale}[2]{\simeqstep{R_{#1} \to #2 R_{#1}}}
\newcommand{\rowAdd}[3]{\simeqstep{R_{#1} \to R_{#1}  #2 R_{#3}}}
\newcommand{\colSwap}[2]{\simeqstep{C_{#1} \leftrightarrow C_{#2}}}
\newcommand{\colScale}[2]{\simeqstep{C_{#1} \to #2 C_{#1}}}
\newcommand{\colAdd}[3]{\simeqstep{C_{#1} \to C_{#1}  #2 C_{#3}}}


\chapternumber{2}
\begin{document}
\makecover{2}

\section{}

\newcommand{\questionA}{
    \mat{-\la+1  & 2\la-1      & \la    \\
        \la      & \la^2       & -\la   \\
        \la^2 +1 & \la^2+\la-1 & -\la^2}
}
\newcommand{\questionB}{\mat{\la^2-1 & 0 \\ 0 & (\la-1)^3}}

求下列$\la -$矩阵的Simth标准型$$(a)\questionA,\quad(b)\questionB$$
\answer
$(a)$
$
    \questionA
    \rowAdd{1}{+}{2}
    \mat{1 & \la^2+2\la-1 & 0\\
        \la & \la^2 & -\la \\
        \la^2+1 & \la^2+\la-1 & -\la^2}
    \simeqstep{}
$

$
    \mat{1 & \la^2+2\la-1 & 0\\
        0 & -\la^3-\la^2+\la & -\la \\
        0 & -\la^4-2\la^3+\la^2-\la & -\la^2}
    \simeqstep{}
    \mat{1 & 0 & 0\\
        0 & -\la^3-\la^2+\la & -\la \\
        0 & -\la^4-2\la^3+\la^2-\la & -\la^2}
    \rowAdd{3}{-\la}{2}
$

$
    \mat{1 & 0 & 0\\
        0 & -\la^3-\la^2+\la & -\la \\
        0 & -\la^3-\la & 0}
    \colSwap{2}{3}
    \mat{1 & 0 & 0\\
        0 & -\la & -\la^3-\la^2+\la\\
        0 & 0 & -\la^3-\la}
    \simeqstep{}
    \mat{1 & 0 & 0\\
        0 & \la & 0\\
        0 & 0 & \la^3+\la}
$\\ \\

$(b)$

$D_1(\la) = \la-1$, $D_2(\la) = (\la+1)(\la-1)^4$

$d_1(\la) = \la-1$, $d_2(\la) = (\la+1)(\la-1)^3$

$\questionB \simeqstep{} \mat{d_1(\la) & 0 \\ 0 & d_2(\la)}$
\qed

\section{}
设$A(\la)$为一个5阶$\la-$矩阵,其秩为4,初等因子为$\la,\la^2,\la^2,\la-1,\la-1,\la+1,(\la+1)^3$,试求$A(\la)$的不变因子及其smith标准形.

\answer
$d_4(\la) = \la^2(\la-1)(\la+1)^3$ \qquad
$d_3(\la) = \la^2(\la-1)(\la+1)$ \qquad
$d_2(\la) = \la$ \qquad
$d_1(\la) = 1$

simth标准形为
$\mr{diag}\left(d_1(\la),d_2(\la),d_3(\la),d_4(\la),0\right)$
\qed

\newpage
\section{}
求下列矩阵的Jordan标准形及其相似变换矩阵$P$.
$$\mat{2 & 1 & 0 & -1 \\ 0 & 2 & 0 & 1 \\ 0 & 0& 2 & 1 \\ 0 & 0 & 0 & 2}$$
\answer
$
    \la I - A =
    \mat{\la-2 & -1 & 0 & 1 \\
        0 & \la-2 & 0 & -1 \\
        0 & 0 & \la-2 & -1 \\
        0 & 0 & 0 & \la-2}
    \simeqstep{}
    \mat{1 & 0 & 0 & 0 \\
        0 & 1 & 0 & 0 \\
        0 & 0 & \la-2 & 0 \\
        0 & 0 & 0 & (\la-2)^3}
$

初等因子为$\la-2,(\la-2)^3$,Jordan标准形为
$$J = \mat{2 & 1 & 0 & 0 \\ 0 & 2 & 1 & 0 \\ 0 & 0 & 2 & 0 \\ 0 & 0 & 0 & 2}$$

设相似变换矩阵为$P=[X_1,X_2,X_3,X_4]$,则$AP=PJ$
$$AP=[AX_1,AX_2,AX_3,AX_4]=PJ=[X_1,X_2,X_3,X_4]J=[2X_1,X_1+2X_2,X_2+2X_3,2X_4]$$
从而有
$$AX_1=2X_1,AX_2=X_1+2X_2,AX_3=X_2+2X_3,AX_4=2X_4$$
首先考虑$(2I-A)X=0$,其全部解为$k_1[1,0,0,0]^\TT+k_2[0,0,1,0]^\TT$

取$X_1=[1,0,0,0]^\TT$,$X_4=[0,0,1,0]^\TT$

随后考虑方程$(A-2I)X=X_1$,其全部解为$[l_1,1,l_2,0]^\TT$,取$X_2=[0,1,1,0]^\TT$

最后考虑方程$(A-2I)X=X_2$,全部解为$[m_1,1,m_2,1]^\TT$,取$X_3=[0,1,0,1]^\TT$

故$P=\mat{1 & 0 & 0 & 0 \\ 0 & 1 & 1 & 0 \\ 0 & 1 & 0 & 1 \\ 0 & 0 & 1 & 0}$
\qed

\newpage
\section{}
已知矩阵$$A=\mat{3 & 0 & 0 \\ a & 3 & 0 \\ c & b & 2}$$求A的所有可能Jordan标准形,并给出A可对角化的条件.

\answer
$|\la E-A|=(\la-2)(\la-3)^2$

特征值$\la_1=2$是一重根,对应一个1阶Jordan块

对于特征值$\la_2=3$,$A-3E=\mat{0 & 0 & 0 \\ -a & 0 & 0 \\ -c & -b &1}$,有如下两种情况:
\begin{itemize}
    \item $a=0$,$\mr{rank}(A-3E)=1$,$q_{\la_2}=2$,对应两个1阶Jordan块.即Jordan标准形为$\mat{3 \\ &3 \\ &&2}$
    \item $a\ne 0$,$\mr{rank}(A-3E)=2$,$q_{\la_2}=1$,对应一个2阶Jordan块.即Jordan标准形为$\mat{3&1 \\ &3 \\ &&2}$
\end{itemize}
$A$可对角化的条件为$a=0$.
\qed

\newpage
\section{}

已知$A^2=E$,证明:$A$相似于矩阵
$$
    \begin{bmatrix}
        1                                \\
         & \ddots                        \\
         &        & 1                    \\
         &        &   & -1               \\
         &        &   &    & \ddots      \\
         &        &   &    &        & -1
    \end{bmatrix}
$$
\answer
设$A$的Jordan标准形为$J=\mr{diag}(J_1,J_2,\cdots,J_n)$,存在可逆矩阵使得$P^{-1}AP=J$,于是有
$$J^2=(P^{-1}AP)^2=P^{-1}A^2P=E$$
即$J_i^2=E_{d_i}$,其中
$$
    J_i=
    \mat{a_i & 1 \\ & a_i & 1 \\ && \ddots & \ddots \\ &&& \ddots & 1 \\ &&&& a_i }_{d_i\times d_i}
    ,\quad
    J_i^2=
    \mat{a_i^2 & a_i & 1 \\ 0 & a_i^2 & a_i & 1 \\ & \ddots & \ddots & \ddots & \ddots\\ && \ddots & \ddots & \ddots & 1\\ &&& \ddots & \ddots & a_i \\ &&&& 0 & a_i^2 }_{d_i\times d_i}
$$
当且仅当$d_i=1$时,才有$J_i^2=E$.经过适当初等变换后,可得

$A$相似于$\mr{diag}(1,\cdots,1,-1,\cdots,-1)$
\qed

\section{}
如果$\la-$矩阵$A(\la)$的秩为3,其初等因子为$\la,\la^2,\la^3,\la+2,(\la+2)^2$,则$A(\la)$的行列式因子为

\answer
$d_3(\la)=\la^3(\la+2)^2$, $d_2(\la)=\la^2(\la+2)$, $d_1(\la)=\la$

$D_3(\la)=\la^6(\la+2)^3$, $D_2(\la)=\la^3(\la+2)$, $D_1(\la)=\la$
\qed

\section{}
$\la-$矩阵$A(\la)=\mat{\la^3(\la-2) \\ & \la(\la+1) \\ && \la(\la-2)^2}$的Smith标准形为

\answer
$D_3(\la)=\la^5(\la+1)(\la-2)^3$, $D_2(\la)=\la^2(\la-2)$, $D_1(\la)=\la$

$d_3(\la)=\la^3(\la-2)^2(\la+1)$, $d_2(\la)=\la(\la-2)$, $d_1(\la)=\la$
\qed

\section{}
下面四个矩阵有三个Jordan块构成的是

\answer
$\mat{-1 & 0 & 0 & 0 \\ 0 & -1 & 1 & 0 \\ 0 & 0 & -1 & 0 \\ 0 & 0 & 0 & -1}$
\qed

\section{}
两个$n$阶复矩阵相似的充要条件为:
\\A. 秩相等
\\B. 初等因子完全相同
\\C. 行列式值相等
\\D. 特征值完全相同

\answer
B\qed

\section{}
将矩阵$A=\mat{1 & -1 & 0 \\ -2 & 2 & 4 \\ 1 & 0 & 3}$表示成一个数量矩阵和一个迹为零的矩阵之和

\answer
$\mat{-1 & -1 & 0 \\ -2 & 0 & 4 \\ 1 & 0 & 1} + \mat{2 & 0 & 0 \\ 0 & 2 & 0 \\ 0 & 0 & 2}$
\qed

\section{}
已知两个$n$维列向量$\alpha,\beta$,且$\beta^\TT\alpha=6$,则矩阵$A=\alpha\beta^\TT$的Jordan标准形为

\answer
$\mr{rank}(A)=1,\mr{tr}(A)=6$, $A$仅存在一个非零特征值为$6$

Jordan标准形为$\mr{diag}(6,0,\cdots,0)$
\qed

\end{document}
